
%
% Commented out this for now, since it would better exist as a
% supplementary section. AES stuff can point at a paper.
%
%\section{Benchmarking}
%
%\subsection{AES}
%\label{sec:benchmark:aes}
%
%The source code for these benchmarks is available as part of the
%{\tt riscv-crypto} 
%\href{https://github.com/scarv/riscv-crypto/tree/master/benchmarks/crypto_block/aes}{GitHub repository}.
%
%\begin{table}[h]
%\centering
%\begin{tabular}{lrrrrr}
%AES Proposal & Instrs Enc & Instrs Dec & Instrs Ks Enc & Instrs Ks Dec & Static Code Size (Bytes) \\ \hline
%Reference   & 3260  & 6704  & 594  &  594  & 6233   \\
%TTable      & 1005  & 1012  & 625  &  1863 & 14353  \\
%V1 Latency  & 2756  & 6537  & 462  &  462  & 3575   \\
%V1 Size     & 2874  & 6681  & 510  &  510  & -      \\
%V2 Latency  & 325   & 307   & 462  &  581  & 1867   \\
%V2 Size     & 552   & 539   & 462  &  687  & -      \\
%V3.1 / RV32 & 321   & 321   & 238  &  682  & 1952   \\
%V3.2        & 480   & 470   & 238  &  826  & 2264   \\
%V4 / RV64   & 101   & 111   & 67   &  132  & 594     \\
%\end{tabular}
%\caption{
%AES instruction execution counts measured using Spike
%running {\tt rv32imcb\_Zscrypto}.
%Instruction counts are for a single AES 128 block encrypt/decrypt operation,
%or for a single encrypt/decrypt key schedule.
%V1 and V2 proposals include results for ``latency" optimised and ``size"
%optimised implementations of the underlying instructions.
%The ``latency" rows assume one cycle per instruction.
%The ``size" rows assume four cycles per instruction (corresponding to
%one implemented SBox), which is modelled by padding each AES instruction
%with 3 additional {\tt nop} instructions.
%Static code size is reported as the sum of all {\tt .text} and {\tt .data}
%sections for encrypt, decrypt and key schedule.
%}
%\label{tab:benchmarks:aes:perf}
%\end{table}
%
%
%\begin{table}[h]
%\centering
%\begin{tabular}{llllllll}
%AES Proposal & Longest Topological Path & NAND2 Gates \\ \hline
%V1 Latency   & 18                       & 2376        \\
%V1 Size      & 21                       & 966         \\
%V2 Latency   & 18                       & 3453        \\
%V2 Size      & 21                       & 1287        \\
%V3.1 / RV32  & 30                       & 1157        \\
%V3.2         & 23                       & 1117        \\
%V4 / RV64 Latency & 28                  & 8482        \\
%\end{tabular}
%\caption{
%AES proposal RTL implementation benchmarks.
%Measured using the Yosys simple CMOS flow.
%The V1 and V2 proposals have "Latency" and "Size" optimised versions.
%The "Latency" versions use $4$ SBox instantiations and compute their
%results in a single clock cycle.
%The "Size" versions use $1$ SBox instantiation, and compute their
%results over 4 clock cycles.
%}
%\label{tab:benchmarks:aes:impl}
%\end{table}
%
%\newpage
%\subsection{SHA2}
%\label{sec:benchmark:sha2}
%
%The source code for the
%SHA256\footnote{\url{https://github.com/scarv/riscv-crypto/tree/master/benchmarks/crypto_hash/sha256}}
%and
%SHA512\footnote{\url{https://github.com/scarv/riscv-crypto/tree/master/benchmarks/crypto_hash/sha512}}
%benchmarks are available as part of the {\tt riscv-crypto}
%\href{https://github.com/scarv/riscv-crypto/tree/master/benchmarks/crypto_block/aes}{GitHub repository}.
%
%\begin{table}[h]
%\centering
%\begin{tabular}{lrrr}
%Architecture      & Static Code Size (Bytes) & Instructions Executed & Performance Gain \\ \hline
%rv32gc            & 14934                    & 78003 & 1.00x          \\
%rv32gcb           & 10086                    & 56866 & 1.37x          \\
%rv32gcb\_Zscrypto & 5938                     & 28539 & 2.73x 
%\end{tabular}
%\caption{{\bf SHA256:}
%Static code size and instructions executed comparison for
%the \mnemonic{ssha256.sx} instructions on RV32 based architectures.
%Instruction execution counts are for hashing 1024 bytes of data using
%SHA256.
%}
%\label{tab:benchmarks:sha256}
%\end{table}
%
%Area and path length estimates are calculated using a basic Yosys CMOS
%synthesis flow.
%
%\begin{table}[h]
%\centering
%\begin{tabular}{lrrr}
%Architecture      & Static Code Size (Bytes) & Instructions Executed & Performance Gain \\ \hline
%rv64gc            & 20490                    & 73138 & 1.00x          \\
%rv64gcb           & 14216                    & 53153 & 1.38x          \\
%rv64gcb\_Zscrypto & 8954                     & 35881 & 2.04x 
%\end{tabular}
%\caption{{\bf SHA512:}
%Static code size and instructions executed comparison for
%the \mnemonic{ssha512.sx} instructions on RV64 based architectures.
%Instruction execution counts are for hashing 1024 bytes of data
%using SHA512.
%}
%\label{tab:benchmarks:sha512}
%\end{table}
%
%\begin{table}[h]
%\centering
%\begin{tabular}{lrr}
%Instructions   & Longest Topological Path & Area (NAND2 Equivalent) \\ \hline
%ssha256.s*     & 5                        & 787                   \\
%ssha512.s*     & 6                        & 1534                  \\
%\end{tabular}
%\caption{
%Area and path length estimates are calculated using a basic Yosys CMOS
%synthesis flow.
%}
%\label{tab:benchmarks:sha2:rtl}
%\end{table}

% ============================================================================

\newpage
\section{Tentative Encoding Proposals}

% NOTE: These are generated by the `bin/parse_opcodes.py` script,
% which is called by $REPO_HOME/doc/Makefile
{\bf Please do not implement these opcodes unless you are happy changing
them in the future!}
\medskip

\import{../build/}{opcodes-crypto.tex}

% ============================================================================

\newpage
\section{Entropy Source: Rationale and Discussion}
\label{sec:entropy-appendix}

    The security of cryptographic systems is based on secret bits and keys.
    To prevent guessing, these bits need to be random, so they come from
    True Random Number Generators (TRNGs).

    As a fundamental security function, the generation of random numbers is
    governed by numerous standards and technical requirements.

\subsection{Standards and Terminology}

    A driving design goal for our architecture was for it to be easy to
    implement, yet compatible with current versions of FIPS 140-3
    \cite{NI19} and NIST SP 800-90B \cite{TuBaKe+18}, significantly
    updated standards that only became requirements in 2020. Naturally,
    the architecture should also support other RNG frameworks such as
    German AIS 20 / AIS 31 \cite{KiSc01,KiSc11,BS13} which is widely used
    in Common Criteria evaluations.

    These standards set many of the technical requirements for the design,
    and we use their terminology if possible.
    The delineation of various components is illustrated in Figure
    \ref{fig:rng_tikz}.


    \paragraph{Entropy Source (ES)}
    \label{sec:intro-es}
    Physical sources of true randomness are called Entropy Sources (ES)
    \cite{TuBaKe+18}. They are built by sampling and processing data
    from a noise source (Section \ref{sec:noise-sources}). Since these
    are directly based on natural phenomena and are subject to
    environmental conditions (which may be adversarial), they require
    features and sensors that monitor the ``health'' and quality of those
    sources.

    \paragraph{Conditioning}
    \label{sec:intro-cond}
    Raw physical randomness (noise) sources are rarely statistically
    perfect and some generate very large amounts of bits, which need to be
    ``debiased'' and reduced to a smaller number of bits. This process is
    called conditioning. A secure hash function is an example of a
    cryptographic conditioner. It is important to note that even though
    hashing may make the output look more random, it does not increase its
    entropy content.

    Non-cryptographic conditioners and extractors such as von Neumann's
    ``debiased coin tossing'' \cite{Ne51} are easier to implement
    efficiently but may reduce entropy content much more than hashes.
    However, they are not based on computational hardness assumptions and
    are therefore inherently more future proof.

    \paragraph{Pseudorandom Number Generator (PRNG)}
    \label{sec:intro-prng}
    Pseudorandom Number Generators (PRNGs) use deterministic mathematical
    formulas to create a large amount of random numbers from a smaller
    amount ``seed'' randomness. PRNGs are divided into cryptographic and
    non-cryptographic ones.

    Non-cryptographic PRNGs, such as the linear-congruential generators
    found in many programming libraries, may generate statistically
    satisfactory random numbers but must never be used for cryptographic
    keying. This is because they are not designed to resist
    \emph{cryptanalysis}; it is usually possible to take some output and
    mathematically derive the ``seed'' or the internal state  of the PRNG
    from it. This is a security problem since knowledge of the state
    allows the attacker to compute future or past outputs.

    \paragraph{Deterministic Random Bit Generator (DRBG)}
    \label{sec:intro-drbg}
    Cryptographic PRNGs are also known as Deterministic Random Bit
    Generators (DRBGs), a term used by SP 800-90A \cite{BaKe15}. A strong
    cryptographic algorithm such as AES \cite{NI01} or SHA-2/3
    \cite{NI15,NI15A} is used to produce random bits from a seed. The secret
    seed material is acting as a cryptographic key; determining the seed
    from DRBG output is as hard as breaking AES or a strong hash function.
    This also illustrates that the seed/key needs to be long enough and
    come from a trusted Entropy Source. The keys can be frequently refreshed
    or ``reseeded''.

\begin{figure}[tb]
    \centering
    %	rng_tikz.tex
%	2020-06-25	Markku-Juhani O. Saarinen <mjos@pqshield.com>

\begin{tikzpicture}

\draw[thick,fill=red!10] (2.0, 0.0) ellipse (1.5 and 0.3) 
	node {\bf Noise Source};

\draw[-Stealth] (2.0,-0.3) -- ++(0.0, -0.6) 
        node[pos=0.5,anchor=east] {\small\em ``analog''}
        node[pos=0.5,anchor=west] {\small\em shot, quantum, ..};

\draw[rounded corners,fill=cyan!10] (2.0+1.2, -0.9) rectangle ++(-2.4,-0.5)
	node[pos=.5] {Sampling};

\draw[-Stealth] (2.0,-1.4) -- ++(0.0, -0.6)
        node[pos=0.5,anchor=east] {\small\em raw bits};

%\draw[rounded corners] (2.0+1.2, -2.0) rectangle ++(-2.4,-0.5)
%	node[pos=.5] {Conditioning};


\draw[thick,fill=cyan!10] (2.0+1.3,-2.0) -- ++(-2.6,-0.0) -- 
       ++(0.2,-0.5) -- ++(2.2,0.0) -- cycle;

\node at (2.0,-2.25) {Conditioning};

\draw[-Stealth, dashed] (2.0,-1.7) -- (4.5-0.8,-1.7);

\draw[-Stealth] (2.0,-2.5) -- ++(0.0, -0.6)
        node[pos=0.5,anchor=east] {\small\em pre-proc.}
        node[pos=0.5,anchor=west] {\small\em {Entropy}$(X) > 8$};

\draw[fill=blue!10] (4.4 - 0.7, -1.7 + 0.45) rectangle ++(1.4,-0.9) 
	node[align=center,pos=.5] {{Health}\\{Tests}};

\draw[-Stealth, dashed] (4.4, -2.15) -- (4.4, -3.35) -- (3.5, -3.35)
        node[pos=0.5,anchor=south] {\tiny\bf OK?};

\draw[thick,fill=green!10] (2.0 - 1.5, -3.1) 
	rectangle ++(3.0,-0.5)  node[pos=.5] {\bf PollEntropy};

\node at (0.0, -3.35) {\bf ISA};

\draw[->, dotted] (0.0,-3.1) -- ++(0.0, 3.3);

\node[align=center,rotate=90] at (0.0,-1.7)
		{\small{\bf Hardware:}\\\small{\em Entropy Source (ES)}};

\draw[->, dotted] (0.0,-3.6) -- ++(0.0, -3.3);

\node[align=center,rotate=90] at (0.0,-5.0)
		{\small{\bf Software:}\\\small{\em Driver + DRBG(s)}};

\draw[-Stealth] (2.0,-3.6) -- ++(0.0, -0.6) 
        node[pos=0.5,anchor=east] {\small\em ( ES16 )}
        node[pos=0.5,anchor=west] {\small\em seed};


\node[align=center] at (3.95,-3.6)
	{\tiny\textcolor{red!50!black}{\underline{\em Not OK:}}};
\node[align=center] at (3.95,-3.9)
	{\tiny\textcolor{red!50!black}{\em ( BIST / WAIT / DEAD )} };


%\draw[rounded corners] (2.0+1.2, -4.2) rectangle ++(-2.4,-0.5)
%	node[pos=.5] {Hash / Pool};


\draw[thick,dashed] (2.0+1.3,-4.2) -- ++(-2.6,-0.0) -- 
       ++(0.2,-0.5) -- ++(2.2,0.0) -- cycle;

\node at (2.0,-4.45) {Entropy Pool};

\draw[-Stealth] (2.0,-4.7) -- ++(0.0, -0.6) 
        node[pos=0.5,anchor=east] {\small\em Processed}
        node[pos=0.5,anchor=west] {\small\em ``full entropy''};


%\draw[thick] (2.0+1.1,-5.3) -- ++(-2.2,-0.0) -- 
%       ++(-0.2,-0.5) -- ++(2.6,0.0) -- cycle;

\draw[thick,fill=brown!10] (2.0+1.2, -5.3) rectangle ++(-2.4,-0.5)
	node[pos=.5] {Crypto DRBG};

\draw[-Stealth] (2.0,-5.8) -- ++(0.0, -0.6) 
        node[pos=0.5,anchor=east] {\small\em Library API}
        node[pos=0.5,anchor=west] {\small\em Kernel syscall};


\draw[thick,fill=gray!20] (2.0, -6.7) ellipse (1.5 and 0.3) 
	node {\bf Application};

\end{tikzpicture}



    \caption{PollEntropy provides an Entropy Source (ES) only, not a stateful
        random number generator. As a result, it can support arbitrary
        security levels. Cryptographic (AES, SHA-2/3) ISA Extension
        instructions can be used to construct high-speed DRBGs that are
        seeded from the entropy source.}
%   \Description{Illustration of PollEntropy at the ISA Boundary.}
    \label{fig:rng_tikz}
\end{figure}

\subsection{Specific Rationale and Considerations}

    \paragraph{(Sect. \ref{sec:es-pollentropy}) PollEntropy:}
    An entropy source does not require a high-bandwidth interface;
    a single high-security DRBG source initialization only requires 256
    bits and the DRBG state can be shared by any number of callers.
    Once initiated, a DRBG requires new randomness only for the purposes of
    forward security.

    Without a polling-style mechanism the entropy source could hang for
    thousands of cycles under some circumstances. The \mnemonic{WFI} mechanism
    (at least potentially) allows energy-saving sleep on MCUs and context
    switching on a higher-end CPUs.

    The reason for the particular \verb|OPST| two-bit mechanism is to
    provide redundancy. The ``fault'' bit combinations 11 (and 00) are more
    likely for electrical reasons if feature discovery fails and the entropy
    source is actually not available (this has happened to AMD \cite{Sa19}).

    The 16-bit bandwidth was a compromise motivated by the desire to
    provide redundancy in the return value, some protection against
    potential Power/DPA/EM leakage (further alleviated by the 2:1
    cryptographic conditioning), and the desire to have all of the
    bits ``in the same place'' on both RV32 and RV64 architectures for
    programming convenience.

    \paragraph{(Sect. \ref{sec:es-polling}) Polling Mechanism:}

    Figure \ref{fig:esstate_tikz} illustrates operational state
    (\verb|OPST|) transitions. The state is usually either WAIT or ES16.
    The baseline specification requires no additional signaling or
    interrupts or to be supported by the hardware. However, the polling
    mechanism should be implemented in a way that allows even generic
    non-interrupt drivers to benefit from interrupts.

\begin{figure}[tb]
    \centering
    %	esstate_tikz.tex
%	2020-06-26	Markku-Juhani O. Saarinen <mjos@pqshield.com>

\begin{tikzpicture}[>=latex,scale=1.2]

\tikzset{every state/.append style={rectangle, rounded corners}}

\node[align=center] (reset) at (0,-0.5) {reset};
\node[state,fill=cyan!10] (bist) at (0,-1.5) {BIST};

\node[state,fill=yellow!20] (wait) at (-1,-3) {WAIT};

\node[state,fill=green!20] (es16) at (1,-3) {ES16};

\node[align=center] (seed) at (2,-3) {{seed}\\{ok!}};

\node[state,fill=red!20] (dead) at (0,-4.5) {DEAD};

\node[align=center] (dead2) at (0,-5.5) {\em stays dead!};

\draw[->] (reset) to (bist);
\draw[->] (bist) to (wait); 
\draw[->] (bist) to (es16); 

\draw[->, bend left] (wait) to (es16); 
\draw[->, bend left] (es16) to (wait);

\draw[dashed,->] (bist) .. controls (-2.5,-2.5) and (-2.5,-3.5) .. (dead); 

\draw[dashed,->] (wait) to (dead); 
\draw[dashed,->] (es16) to (dead);

\draw[->] (dead) .. controls ++(0.15,-0.85) and ++(-0.15,-0.85) .. (dead);

\end{tikzpicture}


    \caption{Normally the operational state alternates between WAIT
        (no data) and ES16, which means that 16 bits of randomness (seed)
        has been polled. DEAD is an unrecoverable state, while BIST
        (Built-in Self-Test) only occurs after reset.}
%   \Description{State diagram of the Entropy Source.}
    \label{fig:esstate_tikz}
\end{figure}


    \paragraph{(Sect. \ref{sec:req-es}) \S E1, Entropy Requirement:}
        Rather than attempting to mathematically define the properties that the
    entropy source output must satisfy, we define that it should be good
    enough to pass an evaluation and certification when conditioned
    cryptographically (``perfectly'') in ratio 2:1. This is our ``safety
    margin'' for non-cryptographic conditioners.

    Note that the min-entropy assessment methodology in SP 800-90B
    \cite{TuBaKe+18} also has a safety margin in its confidence intervals,
    and therefore there must be consistently \emph{more than} 8 bits of
    entropy 16-bit word per byte. In practice, we recommend the
    distribution to be significantly closer to uniform to satisfy additional
    usability and AIS 20 / AIS 31 \cite{KiSc11} requirements.

    Note that the usage of a vetted conditioner (such as SHA-2/3) was
    specified for technical reasons related to SP 800-90B itself;
    non-vetted conditioners may offer similar security.

    The 128-bit output block size was selected because that is the output
    size of the CBC-MAC conditioner specified in \cite{TuBaKe+18} and also
    the smallest key size we expect to see in applications.

    \paragraph{(Sect. \ref{sec:req-es}) \S E2, I.I.D. Requirement:}
    IID is an optional requirement in SP 800-90B \cite{TuBaKe+18} but it
    is needed to prevent information leakage between different entities that
    possibly share the same entropy source. It also significantly
    simplifies certification and vendor-independent driver development.
    The \verb|PollEntity| instruction itself can be later expanded
    to support non-IID sources (e.g. via a different immediate constant).

    \paragraph{(Sect. \ref{sec:req-es}) \S E3, Secret State Requirement:}
    DRBGs can be used to feed other DRBGs but of course that does not
    increase the absolute amount of entropy in the system.
    The entropy source must be able to support current and future security
    standards and applications.

    The 256-bit level is quite arbitrary but maps to
    ``Category 5'' of Post-Quantum Cryptography standards (See section 4.A.5
    ``Security Strength Categories'' in \cite{NI16}) and TOP SECRET schemes
    in Suite B and the newer Governmental CNSA Suite \cite{NS15}.

    \paragraph{(Sect. \ref{sec:security-controls}) Security Controls:}
    Our approach is informed by the experience of designing and implementing
    cryptographic protocols. Some of the most devastating practical attacks
    against real-life cryptosystems have used inconsequential-looking
    additional information, such as padding error messages \cite{BaFoKa+12}
    or timing information \cite{MoSuEi+20}. In cryptography, such
    out-of-band information sources  are called ``oracles''.

    This also applies to the raw noise source. While most developers agree
    that access to the raw noise source would be \emph{nice to have},
    it is less clear why a generic driver would need it, or know what to
    do with it (the raw noise can literally be \emph{anything}).
    Therefore raw source interface has been delegated to an optional
    vendor-specific debug interface in the baseline specification.

    \begin{quote}
    {\it ``The noise source state shall be protected from adversarial
        knowledge or influence to the greatest extent possible. The methods
        used for this shall be documented, including a description of the
        (conceptual) security boundary’s role in protecting the noise source
        from adversarial observation or influence.''}
    \flushright --Noise Source Requirements, NIST SP 800-90B \cite{TuBaKe+18}.
    \end{quote}

    The role of the RISC-V ISA implementation is to try to ensure that the
    hardware-software interface minimizes avenues for adversarial information
    flow; all status information that is unnecessary in normal operation
    should be eliminated. We specifically urge implementors against creating
    unnecessary information flows (``status oracles'') via the custom bits
    or to allow the instruction to disable or affect the TNRG output in any
    significant way. All information flows and interaction mechanisms must
    be considered from an adversarial viewpoint and implemented only if they
    are truly necessary and their security impact can be fully understood.

    We note that the polling interface is not \emph{constant time}.
    However, a polling interface can be modeled as a rejection sampler.
    The timing oracle of a rejection sampler reveals only the statistical
    rejection criteria, which is not secret information in this case.


    \paragraph{(Sect. \ref{sec:security-controls}) \S T1, On-demand testing:}
    Interaction with hardware self-test mechanisms
    from the software side should be minimal; the term ``on-demand'' does not
    mean that the end-user or application program should be able to invoke
    them in the field (the term is a throwback to an age of discrete,
    non-autonomous crypto devices with human operators.)

    \paragraph{(Sect. \ref{sec:security-controls}) \S T2, Continuous checks:}
    Physical attacks can occur while the device is running. The design
    should not try to guide such active attacks by revealing detailed
    status information. Upon detection of an attack the default action
    should be aimed at damage control -- to prevent weak crypto keys from
    being generated.

    There may be requirements for signaling of alarms; AIS 31 specifies
    ``noise alarms'' that can go off with non-negligible probability
    even if the device is functioning correctly; these can't be fatal;
    \verb|WAIT| is sufficient.

    The state of statistical runtime health checks (such as counters)
    is potentially correlated with some secret keying material, hence
    the zeroization requirement.

    \paragraph{(Sect. \ref{sec:security-controls}) \S T3, Fatal error states:}
    These tests can complement other integrity and tamper resistance
    mechanisms (See Chapter 18 of \cite{An20} for examples).

    Some hardware random generators are, by their physical construction,
    exposed to relatively non-adversarial environmental and manufacturing
    issues. However, even such  ``innocent'' failure modes may indicate
    a  \emph{fault attack} \cite{KaScVe13} and therefore should be addressed
    as a system integrity failure rather than as a diagnostic issue.

\subsection{Implementation Strategies}

    As a general rule, RISC-V specifies the ISA only. We provide some
    additional requirements so that portable, vendor-independent middleware
    and kernel components can be created. The actual hardware
    implementation and certification is left to vendors and circuit designers;
    the discussion in this section is purely informational.

    While we do not require entropy source implementations to be
    certified designs, we do expect that they behave in a compatible manner
    and do not create unnecessary security risks to users. Self-evaluation
    and testing following appropriate security standards is usually needed
    to achieve this. NIST has made its SP 800-90B\cite{TuBaKe+18} min-entropy
    estimation package freely available\footnote{EntropyAssessment:
    \url{https://github.com/usnistgov/SP800-90B_EntropyAssessment}} and
    similar free tools are also available\footnote{(In German)
    AIS 31-Implementierung in JAVA:
    \url{https://www.bsi.bund.de/SharedDocs/Downloads/DE/BSI/Zertifizierung/Interpretationen/AIS_31_testsuit_zip}}
    for AIS 31 \cite{KiSc11}.


    \paragraph{Entropy Flow}
    When considering implementation options and trade-offs one must look
    at the entire information flow since each step is interconnected.
    All of these must be implemented:

    \begin{enumerate}
    \item   {\bf A Noise Source} generates private, unpredictable signals
            from stable and well-understood physical random events.
    \item   {\bf Sampling} digitizes the noise signal into a raw stream of
            bits. This raw data also needs to be protected by the design.
    \item   {\bf Continuous health tests} ensure that the noise source
            and its environment meet their operational parameters.
    \item   {\bf Non-cryptographic conditioners} remove much of the bias
            and correlation in input noise: Output entropy $>4$ bits/byte.
    \item   {\bf Cryptographic conditioners} produce nearly full entropy
            output, completely indistinguishable from ideal random.
    \item   {\bf DRBG} takes in $\geq 256$ bits of seed entropy as keying
            material and uses a ``one way'' cryptographic process to rapidly
            generate bits on demand (which do not reveal the seed).
    \end{enumerate}
    Steps 1-4 (possibly 5) are considered to be part of the Entropy
    Source (ES) and provided by the \verb|PollEntropy| instruction.
    Adding the software-side cryptographic steps 5-6 and control logic
    complements it into a True Random Number Generator (TRNG).
    A Linux ``random pool''-type instantiation is illustrated in Figure
    \ref{fig:rng_tikz}.


\subsubsection{Noise Sources}
\label{sec:noise-sources}

    The theory of random signals and electrical noise has been very well
    established since the post - World War II period \cite{Ri44,Ri45,DaRo58}.
    We will give some examples of common noise sources that can be
    implemented in the processor itself (using standard cells).

    \paragraph{Ring Oscillators.}
    The most common entropy source type in production in use today is
    based on ``free running'' ring oscillators and their timing jitter.
    Here an odd number of inverters is connected into a loop from which
    noise source bits are sampled in relation to a reference clock
    \cite{BaLuMi+11}. The sampled bit sequence may be expected to be
    relatively uncorrelated (close to IID) if the sample rate is suitably low
    \cite{KiSc11}. However further processing is usually required.

    AMD \cite{AM17}, ARM \cite{AR17}, and IBM \cite{LiBaBo+13} are
    examples of ring oscillator TRNGs intended for high-security
    applications.
    The differential/feedback Intel construction \cite{HaKoMa12} is slightly
    different but falls into the same general category.

    The main benefits of ring oscillators are: (1) They can be implemented
    with standard cell libraries without external components --
    and even on FPGAs \cite{VaFiAu+10}, (2) there is an established theory
    for their behavior \cite{HaLe98,HaLiLe99,BaLuMi+11}, and (3) ample
    precedent exists for testing and certifying them at the highest security
    levels.

    Ring oscillators also have well-known implementation pitfalls.
    Their output is sometimes highly dependent on temperature,
    which must be taken into account in testing and modeling.
    If the ring oscillator construction is parallelized, it is important
    that the number of stages and/or inverters in each chain is coprime to
    avoid entropy reduction due to harmonic ``Huyghens effect'' \cite{Ba86}.
    Such harmonics can also be inserted maliciously in a frequency
    injection attack, which can have devastating results \cite{MaMo09}.
    Countermeasures are related to circuit design; environmental sensors,
    electrical filters, and usage of a differential oscillator may help.

    \paragraph{Shot Noise.}
    A category of random sources consisting of discrete events
    and modeled as a Poisson process is called ``shot noise''.
    There's a long-established precedent of certifying them; the
    AIS 31 document \cite{KiSc11} itself offers reference designs based on
    noisy diodes. Shot noise sources are often more resistant to
    temperature changes than ring oscillators.
    Some of these generators can also be fully implemented with standard
    cells (The Rambus / Inside Secure generic TRNG IP \cite{Ra20} is described
    as a Shot Noise generator).

    \paragraph{Other types of noise.}
    It may be possible to certify more exotic noise sources and designs,
    although their stochastic model needs to be equally well understood
    and CPU interfaces as secure.
    See Section \ref{sec:quantum} for a discussion of Quantum entropy
    sources.


\subsubsection{Samplers and GetNoise}

    It is necessary to verify that the noise source and sampler output
    matches with their stochastic models. We note that this is
    usually done in a laboratory setting since NIST SP 800-90B \cite{TuBaKe+18}
    urges implementors to protect the source in production devices.
    We are leaving access as a vendor-specific matter and we urge to
    protect the raw source and to make it unavailable to casual users.

    \underline{Rationale:}
    Samplers can generate vast amounts of data. NIST SP 800-90B
    \cite{TuBaKe+18} defines a conceptual interface \verb|GetNoise()|
    for the raw output and also anticipates that the actual
    interfaces ``will depend on the entropy source deployed''.

    \begin{quote}
    \emph{``The vendor may use special methods (or devices, such as an
    oscilloscope) that require detailed knowledge of the source to
    collect raw data. The testing laboratory is required [...] to
    present a rationale why the data collections methods will not alter
    the statistical properties
    of the noise source or explain how to account for any change
    in the source’s statistical characteristics [...]''}
    \flushright -- FIPS 140 Implementation Guidance, 2019 \cite{NICC19}
    \end{quote}

    Building data paths to make the raw noise available through the ISA
    would be problematic as it is unclear how to ``sample''
    possibly up to several gigabits of information per second in a way
    that is appropriately
    representative of its properties. SP 800-90B notes that it is permitted
    that such an interface be available only in
    ``test mode'' and that it is disabled when the source is operational.


\subsubsection{Continuous Health Tests}
\label{sec:cont-tests}

    If NIST SP 800-90B certification is required, the hardware
    should implement at least the health tests defined in Section
    4.4 of \cite{TuBaKe+18}: repetition count test and adaptive
    proportion test.

    Health monitoring requires some state information related
    to noise source to be maintained. The tests should be designed
    in a way that a specific number of samples guarantees a state
    flush (no hung states). We suggest the window size $W=1024$.
    The state is also fully zeroized in a system reset.

    \underline{Rationale:}
    The two mandatory tests can be built with minimal circuitry.
    Full histograms are not required, only simple counter registers:
    repetition count, window count, and sample count.
    Repetition count is reset every time the output sample value
    changes; if the count reaches a certain cutoff limit, a noise alarm
    is signaled. Window counter is used to save every $W$'th
    (typically $W \in { 512, 1024 }$) output. The frequency of this
    reference sample in the following window is counted; cutoff values
    are defined in the standard. We see that the structure of the
    mandatory tests is such that, if well implemented, no information is
    carried beyond a limit of $W$ samples.

    Section 4.5 of \cite{TuBaKe+18} explicitly permits additional
    developer-defined tests and several more were defined in early
    versions of FIPS 140-1 before being ``crossed out''. The choice
    of additional tests depends on the nature and implementation of the
    physical source.

    Especially if a non-cryptographic a conditioner is used in hardware,
    it is possible that the AIS 31 \cite{KiSc11} online tests are
    implemented by driver software. They can also be implemented in hardware.
    For some security profiles AIS 31 mandates that their tolerances are
    set in a way that the probability of an alarm is at least $10^{-6}$
    yearly under ``normal usage''. Such requirements are problematic
    in modern applications since their probability is too high for
    critical systems\footnote{Currently (2020) about $10^{10}$ secure
    elements are shipped yearly, many in critical applications and with
    TRNGs, according to \url{https://www.eurosmart.com}.}.
    There rarely is anything that can or should be done about a non-fatal
    alarm condition in an operator-free, autonomous system. However,
    AIS 31 allows the DRBG component to keep running despite a failure in
    its Entropy Source, so we use a prolonged WAIT (Section
    \ref{sec:security-controls}) to signal a non-fatal statistical error.
    Drivers and applications can react to this appropriately (or simply
    log it) but it will not directly affect the availability of the TRNG
    as a whole.

\subsubsection{Non-cryptographic Conditioners}
\label{sec:noncrypto}

    As noted in Section \ref{sec:intro-cond}, physical randomness sources
    generally require a post-processing step called \emph{conditioning} to
    meet the desired quality requirements, which  are outlined in Section
    \ref{sec:req-es}.

    The approach taken in this interface is to allow a combination of
    non-cryptographic and cryptographic filtering to take place. The
    first stage (hardware) merely needs to be able to distill the entropy
    comfortably above 4 bits per byte, and to guarantee that the samples
    are independent (IID).

    \begin{itemize}
    \item   One may take a set of bits from a noise source and XOR them
            together to produce a less biased (and more independent) bit.
            If the source model is well understood, such a construction
            lends itself well to analysis and entropy estimation \cite{Da02}.
    \item   The von Neumann extractor \cite{Ne51} looks at consecutive
            pairs of bits, rejects 00 and 11, and outputs 0 or 1 for
            01 and 10. It will cut down the number of bits to less than
            25\% but the output is provably unbiased (assuming independence).
    \item   Blum's extractor \cite{Bl86} can be used on sources
            whose behavior resembles $n$-state Markov chains. If its
            assumptions hold, it also removes dependencies, creating an IID
            source.
    \item   Other linear and non-linear correctors such as those
            discussed by Dichtl and Lacharme \cite{La08}.
    \item   One may also implement a full cryptographic conditioner
            in the entropy source hardware, even though the software driver
            is required to implement one too.
    \end{itemize}

    \underline{Rationale:}
    The main advantage of non-cryptographic filters is in their
    energy efficiency, relative simplicity, and amenability to mathematical
    analysis. If well designed, they can be evaluated in
    conjunction with a stochastic model of the noise source itself.
    They do not require computational hardness assumptions.

    In some cases, an entropy source (and the circuit that implements it)
    may have a uniquely identifiable hardware ``signature''. This may be
    useful in some applications (as random sources may exhibit PUF-like
    features) and undesirable in others (anonymized virtualized
    environments and enclaves).
    Such virtualized environments are probably better off just using
    \verb|/dev/urandom| of the host rather than sharing the host's
    hardware-backed Entropy Source to the guest environment. Also note the
    source entropy requirement (Sect. \ref{sec:req-es}, Secret State)
    when sharing such generators.


\subsubsection{Cryptographic Conditioners}
\label{sec:crypto-cond}

    Cryptographic conditioners are always required on the software side of
    the PollEntropy ISA boundary. They may be also implemented on the
    hardware side if necessary. In any case, the PollEntropy output must
    always be compressed 2:1 (or more) before being used as keying material
    or considered ``full entropy''.

    Examples of cryptographic conditioners include the random pool
    of the Linux operating system, secure hash functions (SHA-2/3,
    SHAKE \cite{NI15,NI15A} ),
    and the AES-based CBC-MAC construction of SP 800-90B \cite{TuBaKe+18}.

    In some constructions, such as the Linux RNG and SHA-3/SHAKE \cite{NI15}
    based generators the cryptographic conditioning and output (DRBG)
    generation is provided by the same component.

    \underline{Rationale:}
    For many low-power targets constructions such as Intel's
    \cite{Me18} and AMD's \cite{AM17} hardware AES CBC-MAC conditioner
    would be too complex and expensive to implement solely to serve
    \verb|PollEntropy|. On the other hand, simpler non-cryptographic
    conditioners may be too wasteful on input entropy if very high-quality
    random output is required -- ARM TrustZone TRBG \cite{AR17}
    outputs 10Kbit/sec at 200 MHz. Hence a resource-saving compromise is
    made between hardware and software generation that allows an
    implementation to use the RISC-V cryptographic ISA.

    Even if a DRBG seed obtains a sufficient amount of entropy in total,
    some bits may be more important than others. For example, the IV values
    of a counter-mode DRBG are less important than the key bits;
    if an adversary knows the key bits then the IV (counter value) is
    easy to determine. The inverse is not true. Cryptographic
    conditioning is required to spread the entropy across all bits.

\subsubsection{The Final Random: DRBGs}
\label{sec:drbgs}

    All random bits reaching end users and applications must come from a
    cryptographic DRBG. These are generally implemented by the driver
    component in software, although the RISC-V
    AES and SHA instruction set extensions should be used if available
    \cite{TG20}.

    Currently recommended DRBGs are defined in NIST
    SP 800-90A (Rev 1) \cite{BaKe15}: \verb|CTR_DRBG|, \verb|Hash_DRBG|,
    and  \verb|HMAC_DRBG|. The
    generator must be able to support \verb|security_strength| of
    at least 256. Certification also
    requires known answer tests (KATs) for the symmetric components and
    the DRBG as a whole. These are significantly easier to implement
    in software than in hardware.
    In addition to the directly certifiable SP 800-90A DRBGs, a
    Linux-style random pool construction based on ChaCha20 \cite{Mu20}
    can be used, or an appropriate construction based on SHAKE256
    \cite{NI15}.

    These are just recommendations; programmers can adjust the usage of the
    CPU Entropy Source to meet future requirements.


\subsection{Quantum vs Classical Random}
\label{sec:quantum}

    A Quantum Random Number Generator (QRNG) is a TRNG whose source of
    randomness can be unambiguously identified to be a \emph{specific}
    quantum phenomenon such as quantum state superposition, quantum state
    entanglement, Heisenberg uncertainty, quantum tunneling, spontaneous
    emission, or radioactive decay \cite{IT19}.


    Direct quantum entropy is theoretically the best possible kind of
    entropy. A typical TRNG based on electronic noise is also largely
    based on quantum phenomena and is equally unpredictable - the difference
    is that the relative amount of quantum and classical physics involved is
    difficult to quantify for a classical TRNG.

    QRNGs are designed in a way that allows the amount of quantum-origin
    entropy to be modeled and estimated. This distinction is important in
    the security model used by QKD (Quantum Key Distribution) security
    mechanisms which can be used to protect the physical layer (such as
    fiber optic cables) against interception by using quantum mechanical
    effects directly.

    This security model means that many of the available
    QRNG devices do not use cryptographic conditioning and may fail
    cryptographic statistical requirements \cite{HuHe20}. Hence they should
    perhaps be considered to be entropy sources instead.

    Relatively little research has gone into QNRG implementation security,
    but many QRNG designs are arguably more susceptible to leakage than
    classical generators (such as ring oscillators) as they tend to employ
    external components and mixed materials.

    \begin{quote}
        {\it ``The NCSC believes that classical RNGs will continue to
        meet our needs for government and military applications for the
        foreseeable future.''}
        \flushright -- U.K. QRNG Guidance, March 2020 \cite{NC20}.
    \end{quote}

    \paragraph{Post-Quantum Cryptography.}
    The classical/quantum origin of randomness is not relevant in NIST
    Post-Quantum Cryptography (PQC) \cite{NI16}. Recall that cryptography
    aims to protect the confidentiality and integrity of data itself
    and does not place any requirements on the physical communication
    channel (like QKD). Classical good-quality TRNGs are perfectly fine
    for generating the secret keys for PQC protocols that are hard for
    quantum computers to break, but implementable on classical computers.
    What matters in cryptography is that the secret keys have enough true
    randomness (entropy) and that they are generated and stored securely.

    Of course one must avoid DRBGs that are based on problems that are
    easily solvable with quantum computers, such as factoring \cite{Sh94}
    in the case of Blum-Blum-Shub generator \cite{BlBlSh86}. However
    most symmetric algorithms are less affected as the best quantum
    attacks are still exponential to key size \cite{Gr96}.
%    (rather than polynomial
%   as is the case for RSA and Discrete Logarithms).

    As an example, the original Intel RNG \cite{Me18}, whose output
    generation is based on AES-128 can be attacked using Grover's algorithm
    with approximately square-root effort \cite{JaNaRo+20}.
    While even ``64-bit'' quantum security is extremely difficult to
    break, many applications specify a higher security requirement.
    NIST \cite{NI16} defines AES-128 to be ``Category 1'' equivalent
    post-quantum security, while AES-256 is ``Category 5'' (highest).
    We avoid this possible future issue by exposing a more direct access
    to the entropy source, which can derive its security from
    information-theoretic assumptions only.

%    This eliminates potential problems caused by computational hardness
%   assumptions.



