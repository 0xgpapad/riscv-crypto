
\subsection{Vector Rotation}

These instructions provide bitwise rotation support for the
Vector Extension, analogous to the scalar rotation instructions
in Section \ref{sec:scalar:bitmanip:rotate}.\footnote{
See \url{https://github.com/AndyGlew/Ri5-stuff/wiki/VROT---Vector-Rotate}
for the initial design discussion around vector rotations.
}

\begin{cryptoisa}
vrot.vv     vd, vs2, vs1, vm        // Vector-Vector
    vd.eew[i] = vm[i] ? rotate(EEW,vs2[i], vs1[i]) : 0
\end{cryptoisa}

The vector-vector variant splits \vrd, \vrs{1} and \vrs{2} into
EEW-bit wide elements.
If the corresponding mask bit in \vm is set,
the EEW-bit element in \vrs{2} is rotated
{\em right} by the value in \vrs{1}.
Only the low $log_2(EEW)$ bits of \vrs{1} are used as the rotation
amount; all other bits are {\em ignored}.
If the mask bit in \vm is clear, then the element of \vrd is zeroed.

\question{The zeroing v.s. leave-unmodified semantics of the vector
mask registers are an implementation option. Which should we specify?}

Vector-vector rotation is extremely important to Keccak-based algorithms 
SHA3 and SHAKE.
It is also important to the ChaCha20 stream cipher.

Other forms of rotate instruction include:
vector-scalar (\texttt{.vs}),
vector-immediate (\texttt{.vi})
and
vector-xreg (\texttt{.vx}).
Presently, the vector-vector variant is the most important from a
cryptographic perspective.
We defer requiring other forms until they are required.
