
\subsection{Vector AES Acceleration - Per Round}
\label{sec:vector:aes:single-round}

These instructions each compute a single round of the AES forward
or inverse cipher, or perform one KeySchedule round.
All of the per-round AES instructions use the standard Inverse Cipher
construction \cite[Section 5.3]{nist:fips:197}.
Note this is different to the {\em scalar} AES instructions, which use
the Equivalent Inverse Cipher construction.

\begin{cryptoisa}
// Vector-Scalar
vaese.vs        vrt, vrs1        // Encrypt
vaeselast.vs    vrt, vrs1        // Encrypt - final round
vaesd.vs        vrt, vrs1        // Decrypt
vaesdlast.vs    vrt, vrs1        // Decrypt - final round

// Vector-Vector
vaese.vv        vrt, vrs1        // Encrypt
vaeselast.vv    vrt, vrs1        // Encrypt - final round
vaesd.vv        vrt, vrs1        // Decrypt
vaesdlast.vv    vrt, vrs1        // Decrypt - final round
\end{cryptoisa}

The \mnemonic{vaese.*} and \mnemonic{vaesd.*} instructions take
a single vector source register where each element contains the
current round key (\vrs{1}),
and
a source/destination vector register (\vrt) where each element contains
the current round state.
Using these two inputs, they produce the next encryption/decryption
round state, applying the 
AddRoundKey, (Inverse-)ShiftRows, (Inverse-)SubBytes and (Inverse-)MixColumns
transformations.
The \mnemonic{vaeselast.*} and \mnemonic{vaesdlast.*} instructions
are used for the final round of AES only.
They behave identically to \mnemonic{vaese.*}/\mnemonic{vaesd.*}, but
omit the (Inverse-)MixColumns transformation.

The Vector-Scalar ({\tt *.vs}) instructions use a single round key
contained in element $0$ of {\tt vrs1} for every data element
contained in {\tt vrt}.
The Vector-Vector ({\tt *.vv}) instructions use corresponding
round key and data elements in \vrs{1} and \vrt.

All of these instructions require $\SEW=128$.
Executing these instructions with any other value of \SEW will cause
an invalid opcode exception.
In implementations where $\VLEN<128$, \LMUL may be used to create
vector register groups with enough bits to express $\SEW=128$.
E.g. an implementation with $\VLEN=64$ may set $\LMUL=2$ and $\SEW=128$.

\begin{cryptoisa}
vaes128keyi.vv    vrt,       rnd    // 1  <= rnd <= 10
vaes192keyi.vv    vrt, vrs2, rnd    // 1  <= rnd <= 12
vaes256keyi.vv    vrt, vrs2, rnd    // 2  <= rnd <= 14

vaes128invkeyi.vv vrt,       rnd    // 9  => rnd =>  0
vaes192invkeyi.vv vrt, vrs2, rnd    // 10 => rnd =>  0
vaes256invkeyi.vv vrt, vrs2, rnd    // 12 => rnd =>  0
\end{cryptoisa}

The \mnemonic{vaes*keyi.vv} instructions
are used to compute the {\em next} AES round key for encryption
or decryption.
In the round number is supplied by the $4$-bit {\tt rnd} immediate.
These instructions are all vector-vector instructions.
Setting {\tt vl=1, vstart=0} will cause the instruction to work on only a
single vector element.

The \mnemonic{vaes128keyi.vv} instruction computes
the {\em next} round key
from the
{\em current} round key
stored in \vrt,
and writes the result back to \vrt.

The \mnemonic{vaes192keyi.vv} and \mnemonic{vaes256keyi.vv} instructions
compute the {\em next} round key
from the
{\em current} round key stored in \vrs{1} {\em and}
the {\em previous} round key stored in \vrt.
The {\em next} round key is written back to \vrt.

Each of these instructions requires $\SEW=128$.
Executing these instructions with any other value of \SEW will cause
an invalid opcode exception.
In implementations where $\VLEN<128$, \LMUL may be used to create
vector register groups with enough bits to express $\SEW=128$.

%vaese.vs        vrt, vrs1:
%  vrt[i] = AESEncRnd(vrt[i], vrs1[0])     for i=vstart, i<vl, i++ 
%
%vaeselast.vs    vrt, vrs1:
%  vrt[i] = AESEncLastRnd(vrt[i], vrs1[0]) for i=vstart, i<vl, i++ 
%
%vaesd.vs        vrt, vrs1:
%  vrt[i] = AESDecRnd(vrt[i], vrs1[0])     for i=vstart, i<vl, i++ 
%
%vaesdlast.vs    vrt, vrs1:
%  vrt[i] = AESDecLastRnd(vrt[i], vrs1[0]) for i=vstart, i<vl, i++ 
%
%vaese.vv        vrt, vrs1:
%  vrt[i] = AESEncRnd(vrt[i], vrs1[i])     for i=vstart, i<vl, i++ 
%
%vaeselast.vv    vrt, vrs1:
%  vrt[i] = AESEncLastRnd(vrt[i], vrs1[i]) for i=vstart, i<vl, i++ 
%
%vaesd.vv        vrt, vrs1:
%  vrt[i] = AESDecRnd(vrt[i], vrs1[i])     for i=vstart, i<vl, i++ 
%
%vaesdlast.vv    vrt, vrs1:
%  vrt[i] = AESDecLastRnd(vrt[i], vrs1[i]) for i=vstart, i<vl, i++ 
%
%vaes128keyi     vrt,       rnd
%    vrt[i] = AES128RndKeyNext(vrt[i], rnd) for i=vstart, i<vl, i++
%
%vaes192keyi     vrt, vrs1, rnd
%    vrt[i] = AES192RndKeyNext(vrt[i], vrs1[i], rnd) for i=vstart, i<vl, i++
%
%vaes256keyi     vrt, vrs1, rnd
%    vrt[i] = AES256RndKeyNext(vrt[i], vrs1[i], rnd) for i=vstart, i<vl, i++
%
%vaes128invkeyi  vrt,       rnd
%    vrt[i] = AES128RndKeyPrev(vrt[i], rnd) for i=vstart, i<vl, i++
%
%vaes192invkeyi  vrt, vrs1, rnd
%    vrt[i] = AES192RndKeyPrev(vrt[i], vrs1[i], rnd) for i=vstart, i<vl, i++
%
%vaes256invkeyi  vrt, vrs1, rnd
%    vrt[i] = AES256RndKeyPrev(vrt[i], vrs1[i], rnd) for i=vstart, i<vl, i++

\begin{figure}[h]
\begin{lstlisting}[language=pseudo]
vaese.vs    vrt, vrs1:
    for i=vstart, i<vl, i++:
        tmp.128  = vrt[i]
        tmp.8[j] = AESSBox[tmp.8[j]] for i = 0..15      // SubBytes
        tmp.128  = AESShiftRows(tmp.128)                // ShiftRows
        tmp.128  = AESMixColumns(tmp.128)               // MixColumns
        tmp.128  = tmp.128 ^ vrs1[i]                    // AddRoundKey

vaesd.vs    vrt, vrs1:
    for i=vstart, i<vl, i++:
        tmp.128  = vrt[i]
        tmp.8[j] = InvAESSBox[tmp.8[j]] for i = 0..15   // InvSubBytes
        tmp.128  = InvAESShiftRows(tmp.128)             // InvShiftRows
        tmp.128  = InvAESMixColumns(tmp.128)            // InvMixColumns
        tmp.128  = tmp.128 ^ vrs1[i]                    // InvAddRoundKey

vaeselast.vs    vrt, vrs1:
    for i=vstart, i<vl, i++:
        tmp.128  = vrt[i]
        tmp.8[j] = AESSBox[tmp.8[j]] for i = 0..15      // SubBytes
        tmp.128  = AESShiftRows(tmp.128)                // ShiftRows
        tmp.128  = tmp.128 ^ vrs1[0]                    // AddRoundKey

vaesdlast.vs    vrt, vrs1:
    for i=vstart, i<vl, i++:
        tmp.128  = vrt[i]
        tmp.8[j] = InvAESSBox[tmp.8[j]] for j = 0..15   // InvSubBytes
        tmp.128  = InvAESShiftRows(tmp.128)             // InvShiftRows
        tmp.128  = tmp.128 ^ vrs1[0]                    // InvAddRoundKey
\end{lstlisting}
\caption{Pseudocode for the per-round vector-scalar AES instructions.
The vector-vector versions are identical, save for the final
line of each, where {\tt vrs1[0]} is replaced with {\tt vrs1[i]}.}
\label{fig:pseudo:aes:vector:per-round}
\end{figure}

\begin{figure}[h]
\begin{lstlisting}[language=pseudo]
vaes128keyi.vv  vrt, rnd:
    for i=vstart, i<vl, i++:
        tmp.128       = vrt[i]
        tmp.32[0]    ^= RoundConstants[rnd]
        t1.32         = ROTR32(tmp.32[3], 8)
        t1.8[x]       = AESSBox[t1.8[x]] for x=0..3
        vrt[i].32[0] ^= t1.32
        vrt[i].32[1] ^= vrt[i].32[0]
        vrt[i].32[2] ^= vrt[i].32[1]
        vrt[i].32[3] ^= vrt[i].32[2]
        if(rnd >= 1 and rnd <= 9):                      // Equivalent
            vrt[i] = InvAESMixColumns(vrt[i])           // Inverse Cipher
	
// TBD - Other KeySchedule Instructions.
\end{lstlisting}
\caption{Pseudocode for the per-round KeySchedule AES instructions.}
\label{fig:pseudo:aes:vector:per-round:ks}
\end{figure}


\subsection{Vector AES Acceleration - All Round}
\label{sec:vector:aes:all-round}

\begin{cryptoisa}
// Vector-Scalar
vaese128.vs vrt, vrs1 // Encrypt all rounds,       128 bit cipher Key
vaese192.vs vrt, vrs1 // Encrypt all rounds, 2*SEW 192 bit cipher Key
vaese256.vs vrt, vrs1 // Encrypt all rounds, 2*SEW 256 bit cipher Key
vaesd128.vs vrt, vrs1 // Decrypt all rounds,       128 bit last     round key
vaesd192.vs vrt, vrs1 // Decrypt all rounds, 2*SEW 192 bit last two round keys
vaesd256.vs vrt, vrs1 // Decrypt all rounds, 2*SEW 256 bit last two round keys

// Vector-Vector
vaese128.vv vrt, vrs1 // Encrypt all rounds,       128 bit cipher Key
vaese192.vv vrt, vrs1 // Encrypt all rounds, 2*SEW 192 bit cipher Key
vaese256.vv vrt, vrs1 // Encrypt all rounds, 2*SEW 256 bit cipher Key
vaesd128.vv vrt, vrs1 // Decrypt all rounds,       128 bit last     round key
vaesd192.vv vrt, vrs1 // Decrypt all rounds, 2*SEW 192 bit last two round keys
vaesd256.vv vrt, vrs1 // Decrypt all rounds, 2*SEW 256 bit last two round keys
\end{cryptoisa}

These instructions perform an entire block encryption or decryption
operation for the given AES parameterisation.
The \vrt vector register elements contain $128$-bit blocks
to be encrypted or decrypted.
The \vrs{1} vector register elements contain the initial
cipher key.

All of these instructions require $\SEW=128$.
Executing these instructions with any other value of {\tt SEW} will cause
an invalid opcode exception.
In implementations where $\VLEN<128$, \LMUL may be used to create
vector register groups with enough bits to express $\SEW=128$.

The \mnemonic{vaes[e|d]128.*} instructions perform a complete AES
128 block encrypt/decrypt.
The $128$-bit data block is sourced from \vrt.
The $128$-bit cipher key is sourced from \vrs{1}.
The encrypted/decrypted block is written back to \vrt.

The \mnemonic{vaes[e|d]192.*} instructions perform a complete
AES $192$ encrypt/decrypt.
The $128$-bit data block is sourced from \vrt, and is sourced with
$\EEW=\SEW$ and $\EMUL=\LMUL$.
The $192$ bit cipher keys are sourced from \vrs{1}.
The cipher key is aligned to the least-significant
bits of the element, and the upper bits are ignored.
Note that all of these instructions require $\SEW=128$ when executed.
Hence, \vrs{1} is sourced with $\EEW=2*\SEW$ and $\EMUL=2*\LMUL$.
The encrypted/decrypted block is written back to \vrt with
$\EEW=\SEW$ and $\EMUL=\LMUL$.

The \mnemonic{vaes[e|d]256.*} instructions perform a complete
AES $256$ encrypt/decrypt.
The $128$-bit data block is sourced from \vrt, and is sourced with
$\EEW=\SEW$ and $\EMUL=\LMUL$.
The $256$ bit cipher keys are sourced from \vrs{1}.
Note that all of these instructions require $\SEW=128$ when executed.
Hence, \vrs{1} is sourced with $\EEW=2*\SEW$ and $\EMUL=2*\LMUL$.
The encrypted/decrypted block is written back to \vrt with
$\EEW=\SEW$ and $\EMUL=\LMUL$.

If an implementation does not support the required $\EEW=2*\SEW$
read mechanisms for the AES-192 and AES-256 instructions, then an invalid
opcode exception is raised.

The vector-scalar ({\tt *.vs}) instructions use the same key, stored
in element $0$ of \vrs{1} to encrypt/decrypt every data block in \vrt.
The vector-vector ({\tt *.vv}) instructions use corresponding elements
of \vrs{1} and \vrt to perform encryption/decryption.

\begin{cryptoisa}
vaes128rkey.vv vrt    // Compute final round key from cipher key.
vaes192rkey.vv vrt    // Compute final two round keys from cipher key.
vaes256rkey.vv vrt    // Compute final two round keys from cipher key.
\end{cryptoisa}

These instructions are used to quickly derive the final round key(s)
based on the initial cipher key.
The final round key(s) are then used by the all rounds decryption
instructions.
These instructions are all vector-vector instructions.
Setting {\tt vl=1, vstart=0} will cause the instruction to work on only a
single vector element.

Each of these instructions requires $\SEW=128$
Executing these instructions with any other value of \SEW will cause
an invalid opcode exception.

The \mnemonic{vaes128rkey.vv} instruction takes a single
vector source register \vrt, and for each $128$-bit element, computes
the final AES-128 round key.
The result is then written back to {\tt vrt}.

The \mnemonic{vaes192rkey.vv} instruction takes a single
vector source register {\tt vrt}, and for each $256$-bit element
(containing a $192$-bit cipher key),
computes the final two AES-192 round keys.
The result is then written back to {\tt vrt}.
Note that this instruction requires $\SEW=128$ when executed.
Hence, \vrt is sourced with $\EEW=2*\SEW$ and $\EMUL=2*\LMUL$.
The result is written back to \vrt with
$\EEW=2*\SEW$ and $\EMUL=2*\LMUL$.

The \mnemonic{vaes256rkey.vv} instruction takes a single
vector source register {\tt vrt}, and for each $256$-bit element
(containing a $256$-bit cipher key),
computes the final two AES-256 round keys.
The result is then written back to {\tt vrt}.
Note that this instruction requires $\SEW=128$ when executed.
Hence, \vrt is sourced with $\EEW=2*\SEW$ and $\EMUL=2*\LMUL$.
The result is written back to \vrt with
$\EEW=2*\SEW$ and $\EMUL=2*\LMUL$.

\todo{For the vaes192/256rkey.vv instructions, they essentially operate on
$256$-bit elements. Currently they are defined as needing $\SEW=128$
and then {\em interpreting} \vrt with $\EEW=2*\SEW$.
This keeps them consistent with the all rounds encrypt/decrypt
instructions.
Would it be simpler to simply require that $\SEW=256$ for these
two instructions?}

\begin{figure}[h]
\begin{lstlisting}[language=pseudo]
TBD
\end{lstlisting}
\caption{Pseudocode for the all-round vector AES instructions.}
\label{fig:pseudo:aes:vector:all-round}
\end{figure}

