
\newpage
\subsection{Scalar SHA2 Acceleration}

\begin{cryptoisa}
RV32, RV64:
    ssha256.s0 rd, rs1 : rd = ror32(rs1, 7) ^ ror32(rs1, 18) ^ srl32(rs1, 3)
    ssha256.s1 rd, rs1 : rd = ror32(rs1,17) ^ ror32(rs1, 19) ^ srl32(rs1,10)
    ssha256.s2 rd, rs1 : rd = ror32(rs1, 2) ^ ror32(rs1, 13) ^ ror32(rs1,22)
    ssha256.s3 rd, rs1 : rd = ror32(rs1, 6) ^ ror32(rs1, 11) ^ ror32(rs1,25)

RV64:
    ssha512.s0 rd, rs1 : rd = ror64(rs1, 1) ^ ror64(rs1,  8) ^ srl64(rs1, 7)
    ssha512.s1 rd, rs1 : rd = ror64(rs1,19) ^ ror64(rs1, 61) ^ srl64(rs1, 6)
    ssha512.s2 rd, rs1 : rd = ror64(rs1,28) ^ ror64(rs1, 34) ^ ror64(rs1,39)
    ssha512.s3 rd, rs1 : rd = ror64(rs1,14) ^ ror64(rs1, 18) ^ ror64(rs1,41)
\end{cryptoisa}

The {\tt ssha256.sX}
instructions implement the core of the four sigma and sum functions used in
the SHA256 hash function \cite[Section 4.1.2]{nist:fips:180:4}.
These operations will be supported for a both RV32 and RV64 targets.
For RV32, the entire XLEN source register is operated on.
For RV64, the low 32-bits of the XLEN register are read and operated on,
with the result zero extended to XLEN bits.
Though named for SHA256, the instructions work for both the
SHA-224 and SHA-256 parameterisations as described in
\cite{nist:fips:180:4}.

The \mnemonic{ssha512.sX}
instructions implement the core of the four sigma and sum functions used in
the SHA512 hash function \cite[Section 4.1.3]{nist:fips:180:4}.
These operations will be supported for RV64 targets only.
Though named for the SHA-512 parameterisation, the instructions
can be used for all of the SHA-384, SHA-512, SHA-512/224 and SHA-512/256
parameterisations as described in \cite{nist:fips:180:4}.

Performance, static code size and RTL benchmarks for these
instructions are found in Section
\ref{sec:benchmark:sha2}.
Together, the instructions occupy $8$ encoding points.

\note{
The remaining two core functions which make up the SHA256/512
hash functions are the $Ch$ and $Maj$ functions:
\begin{itemize}
\item \lstinline{Ch(x,y,z)  = (x & y) ^ (~x & z)}
\item \lstinline{Maj(x,y,z) = (x & y) ^ ( x & z) ^ ( y & z )}
\end{itemize}
As ternary functions, they are much too expensive in terms of
opcode space to consider for inclusion as dedicated instructions for
such a specialist use case.
They are amenable however to macro-op fusion on cores which implement it.
}

\note{
The Sigma functions of SHA-256 are identical to the P0 and P1 permutations
used in the SM3 hash function.
Hence, the \mnemonic{ssha256.s[2,3]} instructions can be used for both
algorithms.
}
