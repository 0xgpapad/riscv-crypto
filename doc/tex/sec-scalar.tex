
As per the RISC-V Cryptographic Extensions Task Group charter:
``{\em The committee will also make ISA extension proposals for lightweight
scalar instructions for 32 and 64 bit machines that improve the performance
and reduce the code size required for software execution of common algorithms
like AES and SHA and lightweight algorithms like PRESENT and GOST}".

\bigskip

For context, some these instructions have been developed based on academic
work at the University of Bristol as part of the XCrypto project
\cite{MPP:19},
and work by
Paris Telecom on acceleration of lightweight block ciphers
\cite{TGMGD:19}.

% ============================================================================

\subsection{Shared Bitmanip Extension Functionality}

Many of the primitive operations used in symmetric key cryptography
and cryptographic hash functions are well supported by the
RISC-V Bitmanip \cite{riscv:bitmanip:repo} extension
\footnote{
At the time of writing, the Bitmanip extension is still undergoing
standardisation.
Please refer to the bitmanip draft specification
\cite{riscv:bitmanip:draft}
directly for the
latest information, as it may be slightly ahead of what is described
here.
}.
We propose that the scalar cryptographic extension {\em reuse} a
subset of the instructions from the Bitmanip extension directly.
Specifically, this would mean that
a core implementing
{\em either}
the scalar cryptographic extensions,
{\em or}
the bitmanip extension,
{\em or}
both,
would be able to depend on the existance of these instructions.

\subsubsection{Rotations}

\todo{}


\subsubsection{Other Permutations}

\todo{}


\subsubsection{Carryless Multiply}

\todo{}


\subsubsection{Conditional Move}

\todo{}


\subsubsection{Logic With Negate}

\todo{}


\subsubsection{Packing}

\todo{}


% ============================================================================

\subsection{LUT4 Instruction}

\todo{}

% ============================================================================

\subsection{Multi-precision Arithmetic}

\todo{}

% ============================================================================

\subsection{Lightweight AES Acceleration}

\todo{}

% ============================================================================

\subsection{Lightweight SHA2 Acceleration}

\todo{}

% ============================================================================

\subsection{Lightweight SHA3 Acceleration}

\todo{}

% ============================================================================

\subsection{Micro-architectural Recommendations}

\todo{Macro-op fusion suggestions, side-channel considerations.}

% ============================================================================

