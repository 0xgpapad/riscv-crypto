
As per the RISC-V Cryptographic Extensions Task Group charter:
``{\em The committee will also make ISA extension proposals for lightweight
scalar instructions for 32 and 64 bit machines that improve the performance
and reduce the code size required for software execution of common algorithms
like AES and SHA and lightweight algorithms like PRESENT and GOST}".

\bigskip

For context, some these instructions have been developed based on academic
work at the University of Bristol as part of the XCrypto project
\cite{MPP:19},
and work by
Paris Telecom on acceleration of lightweight block ciphers
\cite{TGMGD:19}.

The scalar cryptography extension is designed with the following
policies in mind.
Their purpose is to help make design decisions consistent across the
extension.
Deviation from these policies is allowed if well justified.

\note{
These policies {\em are up for discussion}.
}

\policy{
Where there is a choice between:
1) supporting diverse implementation strategies for an algorithm
or
2) supporting a single implementation style which is more performant /
   less expensive;
the scalar crypto extension will pick the more constrained but performant
option.
This fits a common pattern in other parts of the RISC-V specification,
where recommended (but not required) instruction sequences for performing
particular tasks are given as an example, such that both hardware and
software implementers can optimise for only a single use case.
}

\policy{
The extension will be designed to well support {\em existing} standardised
cryptographic constructs.
It will not try to support proposed standards, or cryptographic
constructs which exist only in academia.
Cryptographic standards which are settled upon concurrently with, or after
the RISC-V cryptographic extension standardisation will be dealt with
by future additions too, or versions of, the RISC-V cryptographic
standard extension.
}
\footnote{It is anticipated that the NIST Lightweight Cryptography
contest, and possibly the NIST Post-Quantum Cryptography contest
may be dealt with this way, depending on timescales.}

\policy{
Historically, there has been some discussion \cite{LSYRR:04} on
how newly supported operations in general purpose computing might
enable new bases for cryptographic algorithms.
The standard will not try to anticipate new useful low level
operations which {\em may} be useful as building blocks for
future cryptographic constructs.
}

\policy{
Regarding side-channel countermeasures:
Where relevant, proposed instructions must aim to remove the
possibility of any timing side-channels.
For side-channels based on power or electro-magnetic (EM) measurements,
the extension will not aim to support countermeasures which are
implemented above the ISA abstraction layer.
Recommendations will be given where relevant on how micro-architectures
can implement instructions in a power/EM side-channel resistant way.
}


% ============================================================================

\import{./}{sec-scalar-bitmanip.tex}
\import{./}{sec-scalar-lut4.tex}
\import{./}{sec-scalar-mparith.tex}
\import{./}{sec-scalar-aes.tex}
\import{./}{sec-scalar-sha2.tex}
\import{./}{sec-scalar-sha3.tex}
\import{./}{sec-scalar-sm4.tex}
\import{./}{sec-scalar-ildst.tex}

% ============================================================================

\subsection{Micro-architectural Recommendations}

\todo{Macro-op fusion suggestions, side-channel considerations.}

% ============================================================================

