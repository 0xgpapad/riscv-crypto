
The following instructions extend the RISC-V Vector (``V'') extension.
We use the following notation to refer to vector registers:
{\tt vrd}  is a vector        destination register,
{\tt vrsX} is a vector source             register
and
{\tt vrt}  is a vector source/destination register.

The following definitions are taken from the draft
vector specification {\tt v0.9}\footnote{\url{https://github.com/riscv/riscv-v-spec/releases/tag/0.9}}.
They are re-produced here to make the following sections easier to
read using only this document.

\begin{itemize}
\item[\ELEN] - {\em Maximum} size of a single vector element in bits.
              $\ELEN \ge 8$ and must be a power of $2$.
\item[\VLEN] - The width in bits of each vector register.
              $\VLEN \ge \ELEN$ and must be a power of $2$.
\item[\SLEN] - The striping distance in bits.
              $\VLEN \ge \SLEN \ge 32$ and must be a power of $2$.
\item[\SEW]  - The Standard Element Width in bits.
              Each vector register is viewed as $N=\VLEN/\SEW$ elements.
\item[\LMUL] - The Vector Length Multiplier.
              $\LMUL\in\{{1\over 8},{1\over 4},{1\over 2},1,2,4,8\}$.
              When $\LMUL\ge1$, multiple physical vector registers are
              concatenated together to form a single {\em logical} register.
              $\LMUL<1$ is not salient to the vector crypto instructions.
              When $\LMUL>1$, only vector register addresses which are
              integer multiples of $\LMUL$ may be accessed.
\item[\VLMAX]- Is the maximum number of elements a single vector register
              can contain: $\VLMAX=\LMUL*\VLEN/\SEW$.
\item[\EEW]  - Effective Element Width. This is defined per vector register
              operand, and allows for instructions which operate on
              two sets of elements of different widths.
\item[\EMUL] - Effective $\LMUL$. Like $\EEW$, this is defined per vector
              register operand, allowing for widening and narrowing
              operations.
\end{itemize}

The base vector extension has the constraint $\VLEN \ge \ELEN$.
The vector crypto instructions require that $\ELEN \ge 128$ for all
of it's instructions, and upto $1024$ for some.
To widen the set of implementation choices for $\VLEN$ which will support
the vector crypto extension, we allow $\ELEN$ to vary with $\LMUL$,
requiring instead that $\VLEN*\LMUL \ge \ELEN$.

% ============================================================================

\import{./}{sec-vector-aes.tex}
\import{./}{sec-vector-clmul.tex}
\import{./}{sec-vector-sha2.tex}

% ============================================================================
