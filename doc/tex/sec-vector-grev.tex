
\subsection{Vector GREV}

These instructions are vectorised versions of the Bitmanip
\mnemonic{grevi} instruction described in
Section \ref{sec:scalar:bitmanip:grev}.\footnote{
See \url{https://github.com/AndyGlew/Ri5-stuff/wiki/VGREV---Vector-Generalized-Reverse}
for the initial design discussion around vector \mnemonic{grev}.
}

\begin{cryptoisa}
vgrev.vi     vd, vs2, uimm, vm  // Vector-Immediate
    vd.eew[i] = vm[i] ? generalised_reverse(EEW, vs2.eew[i], uimm) : vs2[i]
\end{cryptoisa}

For each \EEW-bit wide element in \vrs{2}, if the corresponding $\vm[i]$
bit is clear, then the destination element in \vrd is {\em zeroed}.
If the $\vm[i]$ bit is set, then we apply the
\texttt{generalised\_reverse} operation to the \EEW-bit element, using the
$6$-bit \texttt{uimm} immediate as the control variable.
For a complete specification of the \texttt{generalised\_reverse}
operation, see 
\cite[Section 2.2.2, Generalized Reverse]{riscv:bitmanip:draft}.

For the Cryptography Extension,
Implementations must support an \EEW upto and including \XLEN.
Executing the instruction with an un-supported \EEW results in an
Invalid Opcode Exception.
Only the values of \texttt{uimm} listed in table \ref{tab:vgrev:uimm}
are required by the Cryptography Extension.
An un-supported value of \texttt{uimm} results in an
Invalid Opcode Exception.

\begin{table}
\centering
\begin{tabular}{ll}
\texttt{uimm} Value & Operation                         \\
\hline
\texttt{0b000111}   & Reverse bits within bytes         \\
\texttt{0b011000}   & Reverse bytes within 32-bit words \\
\texttt{0b111000}   & Reverse bytes within 64-bit words \\
\hline
\end{tabular}
\caption{Required values of \texttt{uimm} which Cryptography Extensions
of \texttt{vgrev.vi} must support.}
\label{tab:vgrev:uimm}
\end{table}
