%\section{The Entropy Source Interface}
\label{sec:entropy-source}

    The RISC-V True Random Number Generator (TRNG) ISA Extension is primarily
    an Entropy Source (ES) interface. A valid implementation should satisfy
    properties that allow it to be used to \emph{seed} standard and
    nonstandard cryptographic DRBGs of virtually any state size and security
    level.

    The purpose of this baseline specification is to guarantee that a simple,
    device-independent driver component (e.g. in Linux kernel, embedded
    firmware, or a cryptographic library) can use the instruction to
    generate truly random bits. See Appendix \ref{sec:entropy-appendix} for
    rationale and further discussion.

\subsection{PollEntropy Instruction}
\label{sec:es-pollentropy}


    The ISA-level interface consists of a single instruction,
    \mnemonic{pollentropy} that returns a 32/64-bit value in a CPU register.
    It is invoked in {\bf Machine Mode}.

\begin{cryptoisa}
pollentropy      rd, imm          // Get randomness. imm=0 for baseline operation
\end{cryptoisa}

%\begin{lstlisting}[language=C]
%int PollEntropy(0);               // C calling convention.
%\end{lstlisting}

    The instruction is {\bf non-blocking} and returns immediately, either
    with two status bits \verb|rd[31:30]|=\verb|OPST| set to
    \verb|ES16| (01), indicating successful polling, or with {\bf no}
    entropy and one of three polling failure statuses \verb|BIST| (00),
    \verb|WAIT| (10), or \verb|DEAD| (11), discussed below.

    \begin{center}
    \begin{tabular}{ccl}
    \toprule
    Bits    & Name  & Description \\
    \midrule
    \verb|rd[63:32]|    & {\it sign\_ext}
            & Bit 31 is sign-extended on RV64. \\
    \verb|rd[31:30]|    & \verb|OPST|
            & Status:   \verb|BIST| (00), \verb|ES16| (01),
                        \verb|WAIT| (10),   \verb|DEAD| (11). \\
    \verb|rd[29:24]|    & {\it reserved}
            & For future use by the RISC-V specification. \\
    \verb|rd[23:16]|    & {\it custom}
            & Reserved for custom and experimental use. \\
    \verb|rd[15: 0]|    & \verb|SEED|
            & 16 bits of randomness, only when \verb|OPST=ES16|.    \\
    \bottomrule
    \end{tabular}
    \end{center}

    The Status Bits at \verb|rd[31:30]|=\verb|OPST|:

\begin{itemize}
    \item[00]   \underline{\tt BIST}
    indicates that Built-In Self-Test ``on-demand'' (BIST) statistical
    testing is being performed. In typical implementations,
    \verb|BIST| will last only a few milliseconds, up to a few hundred.
    If the system returns temporarily to \verb|BIST| from any other state,
    this signals a non-fatal (usually non-actionable) self-test alarm.

    \item[01]   \underline{\tt ES16}
    indicates success; the low bits \verb|rd[15:0]| will have 16 bits
    of randomness which must be guaranteed to have at least 8 bits true
    entropy regardless of implementation. For example, \verb|0x4000ABCD|
    is a valid \verb|ES16| status output on RV32, with \verb|0xABCD| being
    the \verb|seed| value.

    \item[10]   \underline{\tt WAIT}
    means that a sufficient amount of entropy is not yet available.
    This is not an error condition and may (in fact) be more frequent than
    ES16, since true entropy sources may not have very high bandwidth.
    If polling in a loop, we suggest calling WFI (wait for interrupt)
    before the next poll.

    \item[11]   \underline{\tt DEAD}
    is an unrecoverable self-test error. This may indicate a hardware
    fault, a security issue, or (extremely rarely) a type-1
    statistical false positive in the continuous testing procedures.
    Implementations do not need to implement \verb|DEAD| as it may not require
    an end-user notification; an immediate lock-down may be a more
    appropriate response in dedicated security devices.
\end{itemize}

    The sixteen bits of randomness in \verb|rd[15:0]|=\verb|seed| polled
    with \verb|ES16| status {must be cryptographically conditioned
    before they can be used as (up to 8 bits of) keying material.
    See Section \ref{sec:req-es}.

    For the purposes of interoperable, baseline functionality, \verb|imm = 0|.
    Nonzero values of \verb|imm| are reserved for future use; the behavior
    described here may not apply when \verb|imm| $\neq$ 0.


\subsection{Polling Mechanism with WFI}
\label{sec:es-polling}

\begin{figure}[tb]
    \centering
    %	esstate_tikz.tex
%	2020-06-26	Markku-Juhani O. Saarinen <mjos@pqshield.com>

\begin{tikzpicture}[>=latex,scale=1.2]

\tikzset{every state/.append style={rectangle, rounded corners}}

\node[align=center] (reset) at (0,-0.5) {reset};
\node[state,fill=cyan!10] (bist) at (0,-1.5) {BIST};

\node[state,fill=yellow!20] (wait) at (-1,-3) {WAIT};

\node[state,fill=green!20] (es16) at (1,-3) {ES16};

\node[align=center] (seed) at (2,-3) {{seed}\\{ok!}};

\node[state,fill=red!20] (dead) at (0,-4.5) {DEAD};

\node[align=center] (dead2) at (0,-5.5) {\em stays dead!};

\draw[->] (reset) to (bist);
\draw[->] (bist) to (wait); 
\draw[->] (bist) to (es16); 

\draw[->, bend left] (wait) to (es16); 
\draw[->, bend left] (es16) to (wait);

\draw[dashed,->] (bist) .. controls (-2.5,-2.5) and (-2.5,-3.5) .. (dead); 

\draw[dashed,->] (wait) to (dead); 
\draw[dashed,->] (es16) to (dead);

\draw[->] (dead) .. controls ++(0.15,-0.85) and ++(-0.15,-0.85) .. (dead);

\end{tikzpicture}


    \caption{Normally the operational state alternates between WAIT
        (no data) and ES16, which means that 16 bits of randomness (seed)
        has been polled. BIST (Built-in Self-Test) only occurs after reset
        or to signal a non-fatal self-test alarm (if reached after WAIT or
        ES16). DEAD is an unrecoverable error state.}
%   \Description{State diagram of the Entropy Source.}
    \label{fig:esstate_tikz}
\end{figure}

    Figure \ref{fig:esstate_tikz} illustrates operational state
    (\verb|OPST|) transitions. The state is usually either \verb|WAIT| or
    \verb|ES16|. There are no mandatory interrupts. However, the polling
    mechanism should be implemented in a way that allows even generic
    non-interrupt drivers to benefit from interrupts.

    We specifically recommend against busy-loop polling on this instruction
    as it may have relatively low bandwidth. Even though no specific interrupt
    sequence is specified, it is required that the \mnemonic{WFI} (wait for
    interrupt) instruction is available and does not trap on systems that
    implement \mnemonic{pollentropy}. The RISC-V ISA allows \mnemonic{WFI} to be
    implemented as a \mnemonic{NOP}.
    As a minimum requirement for portable drivers, a \verb|WAIT| or
    \verb|BIST| from \mnemonic{pollentropy} should be followed by a WFI before
    another \mnemonic{pollentropy} instruction is issued. There is
    no need to poll after a \verb|DEAD| state.

    To guarantee that no sensitive data is read twice and that different
    callers don't get correlated output, it is suggested that hardware
    implements ``wipe-on-read'' on the randomness pathway during each read
    (successful poll). For the same reasons, only complete and fully
    processed randomness words shall be made available via
    \mnemonic{pollentropy} (no half-conditioned buffers or even full buffers
    in \verb|WAIT| state -- even if they are to be ignored by compliant
    callers).

\subsection{Entropy Source Requirements}
\label{sec:req-es}

    Output \verb|SEED| from \mnemonic{pollentropy} is not necessarily fully
    conditioned randomness due to hardware limitations of smaller,
    low-powered implementations. However minimum requirements are
    defined. Therefore a caller should not use the output directly but poll
    twice the amount of required entropy, cryptographically condition
    (hash) it, and use that to seed a cryptographic DRBG.

    \begin{itemize}

    \item[\S E1]    {\bf Entropy Requirement.}
    Each 16-bit output sample (\verb|SEED|) must have more than 8 bits of
    independent, unpredictable entropy. This minimum requirement is
    satisfied if (in a NIST SP 800-90B \cite{TuBaKe+18} assessment) 128
    bits of output entropy can be obtained from each 256-bit
    ($16 \times 16$-bit) \mnemonic{pollentropy} output sequence via a vetted
    cryptographic conditioning algorithm (see Section 3.1.5.1.2 in
    \cite{TuBaKe+18}).

    Driver developers may make this conservative assumption but are not
    prohibited from using more than twice the number of seed bits relative
    to the desired resulting entropy.

    \item[\S E2]    {\bf I.I.D. Requirement.}
    The output must be \emph{Independent and Identically Distributed}
    (IID), meaning that the output distribution does not change over time
    and that output words do not convey information about each other.
    This requirement is satisfied if the construction of the physical source
    and sampling mechanism suggests nothing against the IID assumption
    and the IID tests in Section 5 of NIST SP 800-90B \cite{TuBaKe+18} are
    consistently passed.

    \item[\S E3]    {\bf Secret State Size Requirement.}
    A \mnemonic{pollentropy} implementation can also output fully conditioned,
    perfectly distributed numbers. However, it is required that if a DRBG is
    used as a source, it must have an internal state with at least 256 bits
    of secret entropy. In general, any implementation of \mnemonic{pollentropy}
    that limits the security strength shall not reduce it to less than
    256 bits.

    \end{itemize}



\subsection{Security Controls (Tests)}
\label{sec:security-controls}

    Security controls are not mandatory for RISC-V but are required for
    security certification.
    The primary purpose of a cryptographic entropy source is to produce
    secret keying material. In almost all cases a hardware entropy source
    must implement appropriate \emph{security controls} to guarantee
    unpredictability, prevent leakage, detect attacks, and to deny
    adversarial control over the entropy output or ts generation mechanism.

    Many of the security controls built into the device are called ``health
    checks''. Health checks can take the form of
    integrity checks, start-up tests, and on-demand tests. These tests can
    be implemented in hardware or firmware; typically both. Several are
    mandated by standards such as NIST SP 800-90B \cite{NI19}. The choice
    of appropriate health tests depends on the certification target,
    system architecture, the threat model, entropy source type, and other
    factors.

    Health checks are not intended for hardware diagnostics but for
    detecting security issues -- hence the default action should be aimed
    at damage control (prevent weak crypto keys from being
    generated). Additional ``debug'' mechanisms for raw noise sources
    or additional control may be implemented if necessary, but those should
    not be available when the device is in production use.

    \begin{itemize}

    \item[\S T1]    {\bf On-demand testing.}
    A sequence of simple tests is invoked via resetting, rebooting, or
    powering-up the hardware (not an ISA signal). The implementation will
    simply return  \verb|BIST| during the initial start-up self-test period;
    in any case, the driver must wait for them to finish before starting
    cryptographic operations. Upon failure the entropy source will enter a
    no-output \verb|DEAD| state.

    \item[\S T2]    {\bf Continuous checks.}
    If an error is detected in continuous tests or environmental sensors,
    the entropy source will enter a no-output state. We define that a
    non-critical alarm is signaled if the entropy source returns
    to \verb|BIST| state from live (\verb|WAIT| or \verb|ES16|) states.
	Such a \verb|BIST| alarm should be latched until polled at least once.    
    
    Critical failures will result in \verb|DEAD| state immediately.
    A hardware-based continuous testing mechanism must not make statistical
    information externally available, and it must be zeroized periodically or
    upon demand via reset, power-up, or similar signal.

    \item[\S T3]{\bf Fatal error states,}
    Since the security of most cryptographic operations depends on the
    entropy source, a system-wide ``default deny'' security policy approach
    is appropriate for most entropy source failures.
    A hardware test failure should result in at least in \verb|DEAD| state
    and possibly reset/halt. The entropy source (or its cryptographic client
    application) \emph{must not} be allowed to run if its secure operation
    can't be guaranteed.

    \end{itemize}

    The statistical nature of some tests makes ``type-1'' false
    positives a possibility. Security architects will understand to use
    permanent or hard-to-recover ``security-fuse'' lockdowns only if the
    threshold of a test is such that the probability of false-positive is
    negligible over the entire device lifetime.




