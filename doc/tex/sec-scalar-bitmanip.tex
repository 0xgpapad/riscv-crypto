
% ============================================================================

\subsection{Shared Bitmanip Extension Functionality}

Many of the primitive operations used in symmetric key cryptography
and cryptographic hash functions are well supported by the
RISC-V Bitmanip \cite{riscv:bitmanip:repo} extension
\footnote{
At the time of writing, the Bitmanip extension is still undergoing
standardisation.
Please refer to the Bitmanip draft specification
\cite{riscv:bitmanip:draft}
directly for the
latest information, as it may be slightly ahead of what is described
here.
}.
We propose that the scalar cryptographic extension {\em reuse} a
subset of the instructions from the Bitmanip extension directly.
Specifically, this would mean that
a core implementing
{\em either}
the scalar cryptographic extensions,
{\em or}
the Bitmanip extension,
{\em or}
both,
would be able to depend on the existence of these instructions.

%
% TODO: Venn diagram of proposed instructions.
%

The following subsections give the assembly syntax of instructions
proposed for inclusion in the scalar crypto extension, along with a
set of use-cases for common algorithms or primitive operations.
For information on the semantics of the instructions, we refer directly
to the Bitmanip draft specification.

\subsubsection{Rotations}

\begin{cryptobitmanipisa}
RV32, RV64:                         RV64 only:
    ror    rd, rs1, rs2                 rorw   rd, rs1, rs2
    rol    rd, rs1, rs2                 rolw   rd, rs1, rs2
    rori   rd, rs1, imm                 roriw  rd, rs1, imm
\end{cryptobitmanipisa}

See \cite[Section 3.1.1]{riscv:bitmanip:draft} for details of
these instructions.
Standard bitwise rotation is a primitive operation in many block ciphers and
hash functions.
It particularly features in the ARX (Add, Rotate, Xor) class of
block ciphers
\footnote{\url{https://www.cosic.esat.kuleuven.be/ecrypt/courses/albena11/slides/nicky_mouha_arx-slides.pdf}}
and stream ciphers.

Algorithms making use of 32-bit rotations:
SHA256, AES (Shift Rows), ChaCha20, SM3.

Algorithms making use of 64-bit rotations:
SHA512, SHA3.

\subsubsection{Other Permutations: {\tt grev} and {\tt shfl}}

\begin{cryptobitmanipisa}
RV32, RV64:                         RV64 only:
    grev rd, rs1, rs2                   grevw rd, rs1, rs2
    grevi rd, rs1, imm                  greviw rd, rs1, imm
\end{cryptobitmanipisa}

The Generalised Reverse ({\tt grev*}) instructions can be used for 
``{\em byte-order swap, bitwise reversal, short-order-swap,
word-order-swap (RV64), nibble-order swap, bitwise reversal in a byte}".
These operations are useful for various permutation operations
needed either by block ciphers and hash-functions directly, or for
endianness correction of data.
Endianness correction is important because
cryptography often occurs in the context of communication, which requires
standardised endianness which may be different from the natural machine
endianness.
This instruction is also useful for getting data into the right form
for being posted to a dedicated accelerator.

\begin{cryptobitmanipisa}
RV32, RV64:                         RV64:
    shfl    rd, rs1, rs2                shflw   rd, rs1, rs2
    unshfl  rd, rs1, rs2                unshflw rd, rs1, rs2
    shfli   rd, rs1, rs2
    unshfli rd, rs1, rs2
\end{cryptobitmanipisa}

The generalised shuffle instructions are useful for implementing
generic bit permutation operations.
Algorithms such as 
DES \footnote{
One might reasonably argue that given the heritage of DES, it's support
shouldn't really be any sort of consideration for a forward looking
ISA like RISC-V.
}
and
PRESENT\cite{block:present} with
irregular / odd permutations are most-likely to benefit from this
instruction.

\subsubsection{Carry-less Multiply}

\begin{cryptobitmanipisa}
RV32, RV64:                         RV64 only:
    clmul rd, rs1, rs2                  clmulw rd, rs1, rs2
    clmulh rd, rs1, rs2                 clmulhw rd, rs1, rs2
    clmulr rd, rs1, rs2                 clmulrw rd, rs1, rs2
\end{cryptobitmanipisa}

See \cite[Section 2.6]{riscv:bitmanip:draft} for details of
this instruction.
As is mentioned there, obvious cryptographic use-cases for carry-less
multiply are for Galois Counter Mode (GCM) block cipher operations
\footnote{\url{https://en.wikipedia.org/wiki/Galois/Counter_Mode}}.
GCM is recommended by NIST as a block cipher mode of operation
\cite{nist:gcm}, and is the only {\em required} mode for the TLS 1.3
protocol.

\note{
    To be used in a cryptographic context, the RISC-V extension
    {\em requires} that these instructions be implemented in
    a constant time fashion.
    That is, their execution latency {\em must not} depend on
    their operands.
}


\subsubsection{Logic With Negate}

\begin{cryptobitmanipisa}
RV32, RV64:
    andn rd, rs1, rs2
     orn rd, rs1, rs2
    xorn rd, rs1, rs2
\end{cryptobitmanipisa}

See \cite[Section 2.1.3]{riscv:bitmanip:draft} for details of
these instructions.
These instructions are useful inside hash functions, block ciphers and
for implementing software based side-channel countermeasures like masking.
The {\tt andn} instruction is also useful for constant time word-select
in systems without the ternary Bitmanip {\tt cmov} instruction.

Useful for:
SHA3 Chi step,
SHA2 Ch step
and
software based power/EM side-channel countermeasures based on masking.

\subsubsection{Packing}

\begin{cryptobitmanipisa}
RV32, RV64:                         RV64: 
    pack   rd, rs1, rs2                 packw  rd, rs1, rs2
    packu  rd, rs1, rs2                 packuw rd, rs1, rs2
    packh  rd, rs1, rs2
\end{cryptobitmanipisa}

See \cite[Section 2.1.4]{riscv:bitmanip:draft} for details of
these instructions.
Some lightweight block ciphers
(e.g. SPARX \cite{DPUVGB:16})
use sub-word data types in their primitives.
The Bitmanip pack instructions are useful for performing rotations on
16-bit data elements.
They are also useful for re-arranging halfwords within words, and
generally getting data into the right shape prior to applying transforms.
This is particularly useful for cryptographic algorithms which pass inputs
around as byte strings, but can operate on words made out of those byte
strings.
This occurs for AES when loading blocks and keys (which may not be
word aligned) into registers to perform the round functions.
See Figure \ref{fig:example:pack} for an example.

\begin{figure}
\begin{subfigure}[t]{0.5\textwidth}
\begin{lstlisting}
packh   a0, a0, a1
packh   a1, a2, a3
pack    a0, a0, a1
.
.
.
\end{lstlisting}
\end{subfigure}
\begin{subfigure}[t]{0.5\textwidth}
\begin{lstlisting}
slli    a1, a1,  8
slli    a2, a2, 16
slli    a3, a3, 24
or      a2, a2, a3
or      a0, a0, a1
or      a0, a0, a2
\end{lstlisting}
\end{subfigure}
\caption{Comparison of packing four bytes loaded into GPRs $a0..a3$ into
a single 32-bit word in $a0$ using the Bitmanip \mnemonic[pack*] instructions
vs. the standard RV32 instructions.}
\label{fig:example:pack}
\end{figure}

Algorithms with sub-word rotations/shifts:
SPARX.

Algorithms benefiting from packing bytes into words:
AES, SHA2, SHA3.

