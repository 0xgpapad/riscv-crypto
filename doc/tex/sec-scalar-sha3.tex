
\newpage
\subsection{Scalar SHA3 Acceleration}
\label{sec:scalar:sha3}

\note{
These instructions may be withdrawn as proposals, since in practice
implementations of the SHA3 round function are usually unrolled,
which removes the use-case for these instructions.
}

\begin{cryptoisa}
RV32, RV64:
    ssha3.xy rd, rs1, rs2 : rd = (( rs1    % 5) + 5(           rs2 % 5)) << 3
    ssha3.x1 rd, rs1, rs2 : rd = (((rs1+1) % 5) + 5(           rs2 % 5)) << 3
    ssha3.x2 rd, rs1, rs2 : rd = (((rs1+2) % 5) + 5(           rs2 % 5)) << 3
    ssha3.x4 rd, rs1, rs2 : rd = (((rs1+4) % 5) + 5(           rs2 % 5)) << 3
    ssha3.yx rd, rs1, rs2 : rd = (( rs2    % 5) + 5((2*rs1 + 3*rs2)% 5)) << 3
\end{cryptoisa}

These instructions accelerate code-dense implementations of the SHA3 secure
hash function \cite{nist:fips:202}.
They work on the low $3$ bits of the input {\tt rs1} and {\tt rs2} registers,
and compute indices into the state array of the round function.
They are designed to replace both canonical implementations of the
{\tt index} function and lookup table based implementations.
In both cases, these instructions offer substantial
improvements in performance, static code size and dynamic instruction
bandwidth.
A longer discussion of the merits of these instructions can
be found in Appendix \ref{sec:appendix:sha3}.

\note{
SHA3 is not yet widely used, especially with respect to SHA2.
This can be expected to change over time, particularly as many
of the NIST Post-Quantum Cryptography candidates use SHA3 as an
underlying construct.
}

\note{
If the auto-aligning indexed load and store instructions are included,
then the auto-aligning component of these instructions
may not be needed.
}

