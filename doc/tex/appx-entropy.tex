
\newpage
\section{Entropy Source: Rationale and Discussion}
\label{sec:entropy-appendix}


    The security of cryptographic systems is based on secret bits and keys.
    To prevent guessing, these bits need to be random, so they come from
    True Random Number Generators (TRNGs).

    As a fundamental security function, the generation of random numbers is
    governed by numerous standards and technical requirements.

\subsection{Standards and Terminology}

    A driving design goal for our architecture was for it to be easy to
    implement, yet compatible with current versions of FIPS 140-3
    \cite{NI19} and NIST SP 800-90B \cite{TuBaKe+18}, significantly
    updated standards that are only coming into use in 2020. Naturally,
    the architecture should also support other RNG frameworks such as
    German AIS 20 / 31 \cite{KiSc01,KiSc11} which is widely used
    in Common Criteria evaluations.

    These standards set many of the technical requirements for the design,
    and we use their terminology if possible. Note that FIPS 140-3 /
    SP 800-90B (our main target) imposes requirements on min-entropy, while
    AIS-31 discusses Shannon entropy as well.

    These standards set many of the technical requirements for the design,
    and we use their terminology if possible.
    The delineation of various components is illustrated in Figure
    \ref{fig:rng_tikz}.


    \subsubsection{Entropy Source (ES)}
    \label{sec:intro-es}
    Physical sources of true randomness are called Entropy Sources (ES)
    \cite{TuBaKe+18}. They are built by sampling and processing data
    from a noise source (Section \ref{sec:noise-sources}). Since these
    are directly based on natural phenomena and are subject to
    environmental conditions (which may be adversarial), they require
    features and sensors that monitor the ``health'' and quality of those
    sources. See Section \ref{sec:security-controls} for a discussion about
    such security controls.

    \subsubsection{Conditioning}
    \label{sec:intro-cond}
    Raw physical randomness (noise) sources are rarely statistically
    perfect and some generate very large amounts of bits, which need to be
    ``debiased'' and reduced to a smaller number of bits. This process is
    called conditioning. A secure hash function is an example of a
    cryptographic conditioner. It is important to note that even though
    hashing may make the output look more random, it does not increase its
    entropy content.

    Non-cryptographic conditioners and extractors such as von Neumann's
    ``debiased coin tossing'' \cite{Ne51} are easier to implement
    efficiently but may reduce entropy content (in individual bits removed)
    more than cryptographic hashes which mix the input entropy very efficiently.
    However, they are not based on computational hardness assumptions and
    are therefore inherently more future proof. See Section
    \ref{sec:noncrypto} for a more detailed discussion.

    \subsubsection{Pseudorandom Number Generator (PRNG)}
    \label{sec:intro-prng}
    Pseudorandom Number Generators (PRNGs) use deterministic mathematical
    formulas to create a large amount of random numbers from a smaller
    amount of ``seed'' randomness. PRNGs are divided into cryptographic and
    non-cryptographic ones.

    Non-cryptographic PRNGs, such as the linear-congruential generators
    found in many programming libraries, may generate statistically
    satisfactory random numbers but must never be used for cryptographic
    keying. This is because they are not designed to resist
    \emph{cryptanalysis}; it is usually possible to take some output and
    mathematically derive the ``seed'' or the internal state  of the PRNG
    from it. This is a security problem since knowledge of the state
    allows the attacker to compute future or past outputs.

    \subsubsection{Deterministic Random Bit Generator (DRBG)}
    \label{sec:intro-drbg}
    Cryptographic PRNGs are also known as Deterministic Random Bit
    Generators (DRBGs), a term used by SP 800-90A \cite{BaKe15}. A strong
    cryptographic algorithm such as AES \cite{NI01} or SHA-2/3
    \cite{NI15,NI15A} is used to produce random bits from a seed. The secret
    seed material is like a cryptographic key; determining the seed
    from the DRBG output is as hard as breaking AES or a strong hash function.
    This also illustrates that the seed/key needs to be long enough and
    come from a trusted Entropy Source. The DRBG should still be frequently
    refreshed (reseeded) for forward and backward security.

\begin{figure}[tb]
    \centering
    %	rng_tikz.tex
%	2020-06-25	Markku-Juhani O. Saarinen <mjos@pqshield.com>

\begin{tikzpicture}

\draw[thick,fill=red!10] (2.0, 0.0) ellipse (1.5 and 0.3) 
	node {\bf Noise Source};

\draw[-Stealth] (2.0,-0.3) -- ++(0.0, -0.6) 
        node[pos=0.5,anchor=east] {\small\em ``analog''}
        node[pos=0.5,anchor=west] {\small\em shot, quantum, ..};

\draw[rounded corners,fill=cyan!10] (2.0+1.2, -0.9) rectangle ++(-2.4,-0.5)
	node[pos=.5] {Sampling};

\draw[-Stealth] (2.0,-1.4) -- ++(0.0, -0.6)
        node[pos=0.5,anchor=east] {\small\em raw bits};

%\draw[rounded corners] (2.0+1.2, -2.0) rectangle ++(-2.4,-0.5)
%	node[pos=.5] {Conditioning};


\draw[thick,fill=cyan!10] (2.0+1.3,-2.0) -- ++(-2.6,-0.0) -- 
       ++(0.2,-0.5) -- ++(2.2,0.0) -- cycle;

\node at (2.0,-2.25) {Conditioning};

\draw[-Stealth, dashed] (2.0,-1.7) -- (4.5-0.8,-1.7);

\draw[-Stealth] (2.0,-2.5) -- ++(0.0, -0.6)
        node[pos=0.5,anchor=east] {\small\em pre-proc.}
        node[pos=0.5,anchor=west] {\small\em {Entropy}$(X) > 8$};

\draw[fill=blue!10] (4.4 - 0.7, -1.7 + 0.45) rectangle ++(1.4,-0.9) 
	node[align=center,pos=.5] {{Health}\\{Tests}};

\draw[-Stealth, dashed] (4.4, -2.15) -- (4.4, -3.35) -- (3.5, -3.35)
        node[pos=0.5,anchor=south] {\tiny\bf OK?};

\draw[thick,fill=green!10] (2.0 - 1.5, -3.1) 
	rectangle ++(3.0,-0.5)  node[pos=.5] {\bf PollEntropy};

\node at (0.0, -3.35) {\bf ISA};

\draw[->, dotted] (0.0,-3.1) -- ++(0.0, 3.3);

\node[align=center,rotate=90] at (0.0,-1.7)
		{\small{\bf Hardware:}\\\small{\em Entropy Source (ES)}};

\draw[->, dotted] (0.0,-3.6) -- ++(0.0, -3.3);

\node[align=center,rotate=90] at (0.0,-5.0)
		{\small{\bf Software:}\\\small{\em Driver + DRBG(s)}};

\draw[-Stealth] (2.0,-3.6) -- ++(0.0, -0.6) 
        node[pos=0.5,anchor=east] {\small\em ( ES16 )}
        node[pos=0.5,anchor=west] {\small\em seed};


\node[align=center] at (3.95,-3.6)
	{\tiny\textcolor{red!50!black}{\underline{\em Not OK:}}};
\node[align=center] at (3.95,-3.9)
	{\tiny\textcolor{red!50!black}{\em ( BIST / WAIT / DEAD )} };


%\draw[rounded corners] (2.0+1.2, -4.2) rectangle ++(-2.4,-0.5)
%	node[pos=.5] {Hash / Pool};


\draw[thick,dashed] (2.0+1.3,-4.2) -- ++(-2.6,-0.0) -- 
       ++(0.2,-0.5) -- ++(2.2,0.0) -- cycle;

\node at (2.0,-4.45) {Entropy Pool};

\draw[-Stealth] (2.0,-4.7) -- ++(0.0, -0.6) 
        node[pos=0.5,anchor=east] {\small\em Processed}
        node[pos=0.5,anchor=west] {\small\em ``full entropy''};


%\draw[thick] (2.0+1.1,-5.3) -- ++(-2.2,-0.0) -- 
%       ++(-0.2,-0.5) -- ++(2.6,0.0) -- cycle;

\draw[thick,fill=brown!10] (2.0+1.2, -5.3) rectangle ++(-2.4,-0.5)
	node[pos=.5] {Crypto DRBG};

\draw[-Stealth] (2.0,-5.8) -- ++(0.0, -0.6) 
        node[pos=0.5,anchor=east] {\small\em Library API}
        node[pos=0.5,anchor=west] {\small\em Kernel syscall};


\draw[thick,fill=gray!20] (2.0, -6.7) ellipse (1.5 and 0.3) 
	node {\bf Application};

\end{tikzpicture}



    \caption{PollEntropy provides an Entropy Source (ES) only, not a stateful
        random number generator. As a result, it can support arbitrary
        security levels. Cryptographic (AES, SHA-2/3) ISA Extension
        instructions can be used to construct high-speed DRBGs that are
        seeded from the entropy source.}
%   \Description{Illustration of PollEntropy at the ISA Boundary.}
    \label{fig:rng_tikz}
\end{figure}

\subsection{Specific Rationale and Considerations}

    \paragraph{(Sect. \ref{sec:es-pollentropy}) PollEntropy:}
    An entropy source does not require a high-bandwidth interface;
    a single DRBG source initialization only requires 512 bits
    (256 bits of entropy) and the DRBG state can be shared by any number of
    callers. Once initiated, a DRBG requires new randomness only for the
    purposes of forward security.

    Without a polling-style mechanism the entropy source could hang for
    thousands of cycles under some circumstances. The \mnemonic{wfi} mechanism
    (at least potentially) allows energy-saving sleep on MCUs and context
    switching on a higher-end CPUs.

    The reason for the particular \verb|OPST| two-bit mechanism is to
    provide redundancy. The ``fault'' bit combinations 11 (and 00) are more
    likely for electrical reasons if feature discovery fails and the entropy
    source is actually not available (this has happened to AMD \cite{Sa19}).

    The 16-bit bandwidth was a compromise motivated by the desire to
    provide redundancy in the return value, some protection against
    potential Power/EM leakage (further alleviated by the 2:1 cryptographic
    conditioning discussed in Section \ref{sec:req-entropy}), and the desire
    to have all of the bits ``in the same place'' on both RV32 and RV64
    architectures for programming convenience.

    \paragraph{(Sect. \ref{sec:req-es}) \S E1, Entropy Requirement:}
    Rather than attempting to mathematically define the properties that the
    entropy source output must satisfy, we define that it should
    pass SP 800-90B evaluation and certification when conditioned
    cryptographically (``perfectly'') in ratio 2:1. This is our ``safety
    margin'' for non-cryptographic conditioners.

    Note that the min-entropy assessment methodology in SP 800-90B
    \cite{TuBaKe+18} also has a safety margin in its confidence intervals,
    and therefore there must be consistently \emph{more than} 8 bits of
    entropy per 16-bit word. In practice, we recommend the
    distribution to be significantly closer to uniform to satisfy
    possible additional use cases and AIS 20 / 31 \cite{KiSc11}
    requirements (if those can't be met with a software conditioner).

    Note that the usage of a vetted conditioner (such as SHA-2/3) was
    specified for technical reasons related to SP 800-90B itself;
    non-vetted conditioners may offer similar security.

    The 128-bit output block size was selected because that is the output
    size of the CBC-MAC conditioner specified in \cite{TuBaKe+18} and also
    the smallest key size we expect to see in applications.

    \paragraph{(Sect. \ref{sec:req-es}) \S E2, I.I.D. Requirement:}
    IID is an optional requirement in SP 800-90B \cite{TuBaKe+18} but it
    is needed to prevent information leakage between different entities that
    possibly share the same entropy source. It also significantly
    simplifies certification and vendor-independent driver development.
    The \mnemonic{pollentropy} instruction itself can be later expanded
    to support non-IID sources (e.g. via a different immediate constant).

    \paragraph{(Sect. \ref{sec:req-es}) \S E3, Secret State Requirement:}
    DRBGs can be used to feed other (virtual) DRBGs but that does not
    increase the absolute amount of entropy in the system.
    The entropy source must be able to support current and future security
    standards and applications. The 256-bit requirement maps to
    ``Category 5'' of NIST Post-Quantum Cryptography (4.A.5
    ``Security Strength Categories'' in \cite{NI16}) and TOP SECRET schemes
    in Suite B and the newer U.S. Government CNSA Suite \cite{NS15}.

    {\em Source anonymization.}
    In some cases, an entropy source (or the circuit that interfaces it)
    may have a uniquely identifiable hardware ``signature.'' This can be
    harmless or even useful in some applications (as random sources may
    exhibit PUF-like features) but highly undesirable in others (anonymized
    virtualized environments and enclaves). A DRBG masks such
    statistical features.

    \paragraph{(Sect. \ref{sec:security-controls}) Security Controls:}
    Our approach is informed by the experience of designing and implementing
    cryptographic protocols. Some of the most devastating practical attacks
    against real-life cryptosystems have used inconsequential-looking
    additional information, such as padding error messages \cite{BaFoKa+12}
    or timing information \cite{MoSuEi+20}. In cryptography, such
    out-of-band information sources  are called ``oracles.''

    This also applies to the raw noise source. While most developers agree
    that access to the raw noise source would be \emph{nice to have},
    it is less clear why a generic driver would need it, or know what to
    do with it (the raw noise can literally be \emph{anything}).
    Therefore raw source interface has been delegated to an optional
    vendor-specific debug interface outside the scope of the baseline
    specification.

    \begin{quote}
    {\it ``The noise source state shall be protected from adversarial
        knowledge or influence to the greatest extent possible. The methods
        used for this shall be documented, including a description of the
        (conceptual) security boundary’s role in protecting the noise source
        from adversarial observation or influence.''}
    \flushright --Noise Source Requirements, NIST SP 800-90B \cite{TuBaKe+18}.
    \end{quote}

    The role of the RISC-V ISA implementation is to try to ensure that the
    hardware-software interface minimizes avenues for adversarial information
    flow; all status information that is unnecessary in normal operation
    should be eliminated. We specifically urge implementers against creating
    unnecessary information flows (``status oracles'') via the custom bits
    or to allow the instruction to disable or affect the TRNG output in any
    significant way. All information flows and interaction mechanisms must
    be considered from an adversarial viewpoint and implemented only if they
    are truly necessary and their security impact can be fully understood.

    For example, the entropy polling interface may not be ``constant time.''
    The polling mechanism can be modeled as a rejection sampler; such a
    timing oracle can reveal information about the noise source and the
    rejection criteria, but usually not the random output itself.
    If these are correlated, additional countermeasures are necessary.

    \paragraph{(Sect. \ref{sec:security-controls}) \S T1, On-demand testing:}
    Interaction with hardware self-test mechanisms
    from the software side should be minimal; the term ``on-demand'' does not
    mean that the end-user or application program should be able to invoke
    them in the field (the term is a throwback to an age of discrete,
    non-autonomous crypto devices with human operators.)

    \paragraph{(Sect. \ref{sec:security-controls}) \S T2, Continuous checks:}
    Physical attacks can occur while the device is running. The design
    should avoid guiding such active attacks by revealing detailed
    status information. Upon detection of an attack the default action
    should be aimed at damage control -- to prevent weak crypto keys from
    being generated.

    The statistical nature of some tests makes ``type-1'' false
    positives a possibility. There may also be requirements for signaling
    of non-fatal alarms; AIS 31 specifies ``noise alarms'' that can go off
    with non-negligible probability even if the device is functioning
    correctly; these can be signaled with \verb|BIST|.
    There rarely is anything that can or should be done about a non-fatal
    alarm condition in an operator-free, autonomous system.

    The state of statistical runtime health checks (such as counters)
    is potentially correlated with some secret keying material, hence
    the zeroization requirement.

    \paragraph{(Sect. \ref{sec:security-controls}) \S T3, Fatal error states:}
    These tests can complement other integrity and tamper resistance
    mechanisms (See Chapter 18 of \cite{An20} for examples).

    Some hardware random generators are, by their physical construction,
    exposed to relatively non-adversarial environmental and manufacturing
    issues. However, even such  ``innocent'' failure modes may indicate
    a  \emph{fault attack} \cite{KaScVe13} and therefore should be addressed
    as a system integrity failure rather than as a diagnostic issue.

    Security architects will understand to use
    permanent or hard-to-recover ``security-fuse'' lockdowns only if the
    threshold of a test is such that the probability of false-positive is
    negligible over the entire device lifetime.


\subsection{Implementation Strategies}

    When considering implementation options and trade-offs one must look
    at the entire information flow since each step is interconnected.

    \begin{enumerate}
    \item   {\bf A Noise Source} generates private, unpredictable signals
            from stable and well-understood physical random events.
    \item   {\bf Sampling} digitizes the noise signal into a raw stream of
            bits. This raw data also needs to be protected by the design.
    \item   {\bf Continuous health tests} ensure that the noise source
            and its environment meet their operational parameters.
    \item   {\bf Non-cryptographic conditioners} remove much of the bias
            and correlation in input noise: Output entropy $\gg 4$ bits/byte.
    \item   {\bf Cryptographic conditioners} produce nearly full entropy
            output, completely indistinguishable from ideal random.
    \item   {\bf DRBG} takes in $\geq 256$ bits of seed entropy as keying
            material and uses a ``one way'' cryptographic process to rapidly
            generate bits on demand (without revealing the seed/state).
    \end{enumerate}
    Steps 1-4 (possibly 5) are considered to be part of the Entropy
    Source (ES) and provided by the \mnemonic{pollentropy} instruction.
    Adding the software-side cryptographic steps 5-6 and control logic
    complements it into a True Random Number Generator (TRNG).
%   This information flow is illustrated by Figure \ref{fig:rng_tikz}.

    As a general rule, RISC-V specifies the ISA only. We provide some
    additional requirements so that portable, vendor-independent middleware
    and kernel components can be created. The actual hardware
    implementation and certification is left to vendors and circuit designers;
    the discussion in this section is purely informational.

    While we do not require entropy source implementations to be
    certified designs, we do expect that they behave in a compatible manner
    and do not create unnecessary security risks to users. Self-evaluation
    and testing following appropriate security standards is usually needed
    to achieve this. NIST has made its SP 800-90B\cite{TuBaKe+18} min-entropy
    estimation package freely available\footnote{EntropyAssessment:
    \url{https://github.com/usnistgov/SP800-90B_EntropyAssessment}} and
    similar free tools are also available\footnote{(In German)
    AIS 31-Implementierung in JAVA:
    \url{https://www.bsi.bund.de/SharedDocs/Downloads/DE/BSI/Zertifizierung/Interpretationen/AIS_31_testsuit_zip}}
    for AIS 31 \cite{KiSc11}.

\subsubsection{Noise Sources}
\label{sec:noise-sources}

    The theory of random signals and electrical noise became well
    established in the post-World War II period \cite{Ri44,Ri45,DaRo58}.
    We will give some examples of common noise sources that can be
    implemented in the processor itself (using standard cells).

    \paragraph{Ring Oscillators.}
    The most common entropy source type in production use today is
    based on ``free running'' ring oscillators and their timing jitter.
    Here, an odd number of inverters is connected into a loop from which
    noise source bits are sampled in relation to a reference clock
    \cite{BaLuMi+11}. The sampled bit sequence may be expected to be
    relatively uncorrelated (close to IID) if the sample rate is suitably low
    \cite{KiSc11}. However further processing is usually required.

    AMD \cite{AM17}, ARM \cite{AR17}, and IBM \cite{LiBaBo+13} are
    examples of ring oscillator TRNGs intended for high-security
    applications.

    There are related metastability-based generator designs such as
    Transition Effect Ring Oscillator (TERO) \cite{VaDr10}.
    The differential/feedback Intel construction \cite{HaKoMa12} is slightly
    different but also falls into the same general metastable
    oscillator-based category.

    The main benefits of ring oscillators are: (1) They can be implemented
    with standard cell libraries without external components --
    and even on FPGAs \cite{VaFiAu+10}, (2) there is an established theory
    for their behavior \cite{HaLe98,HaLiLe99,BaLuMi+11}, and (3) ample
    precedent exists for testing and certifying them at the highest security
    levels.

    Ring oscillators also have well-known implementation pitfalls.
    Their output is sometimes highly dependent on temperature,
    which must be taken into account in testing and modeling.
    If the ring oscillator construction is parallelized, it is important
    that the number of stages and/or inverters in each chain is coprime to
    avoid entropy reduction due to harmonic ``Huyghens synchronization''
    \cite{Ba86}.
    Such harmonics can also be inserted maliciously in a frequency
    injection attack, which can have devastating results \cite{MaMo09}.
    Countermeasures are related to circuit design; environmental sensors,
    electrical filters, and usage of a differential oscillator may help.

    \paragraph{Shot Noise.}
    A category of random sources consisting of discrete events
    and modeled as a Poisson process is called ``shot noise.''
    There's a long-established precedent of certifying them; the
    AIS 31 document \cite{KiSc11} itself offers reference designs based on
    noisy diodes. Shot noise sources are often more resistant to
    temperature changes than ring oscillators.
    Some of these generators can also be fully implemented with standard
    cells (The Rambus / Inside Secure generic TRNG IP \cite{Ra20} is
    described as a Shot Noise generator).

    \paragraph{Other types of noise.}
    It may be possible to certify more exotic noise sources and designs,
    although their stochastic model needs to be equally well understood
    and their CPU interfaces must be secure.
    See Section \ref{sec:quantum} for a discussion of Quantum entropy
    sources.


\subsubsection{Samplers and GetNoise}

    It is necessary to verify that the noise source and sampler output
    matches with their stochastic models. This is usually
    done in a laboratory setting since NIST SP 800-90B \cite{TuBaKe+18}
    requires that the noise source in protected in production devices.
    We are leaving access as a vendor-specific matter but we urge them to
    protect the raw source and to make it unavailable to casual users.

    \underline{Rationale:}
    Samplers can generate vast amounts of data. NIST SP 800-90B
    \cite{TuBaKe+18} defines a conceptual interface \verb|GetNoise()|
    for the raw output and also anticipates that the actual
    interfaces ``will depend on the entropy source deployed.''

    Building data paths to make the raw noise available through the ISA
    would be problematic as it is unclear how to ``sample''
    possibly up to several gigabits of information per second in a way
    that is appropriately representative of its properties.

    \begin{quote}
    \emph{``The vendor may use special methods (or devices, such as an
    oscilloscope) that require detailed knowledge of the source to
    collect raw data. The testing laboratory is required [...] to
    present a rationale why the data collections methods will not alter
    the statistical properties
    of the noise source or explain how to account for any change
    in the source’s statistical characteristics [...]''}
    \flushright -- FIPS 140 Implementation Guidance, 2020 \cite{NICC20}
    \end{quote}



\subsubsection{Continuous Health Tests}
\label{sec:cont-tests}

    If NIST SP 800-90B certification is required, the hardware
    should implement at least the health tests defined in Section
    4.4 of \cite{TuBaKe+18}: repetition count test and adaptive
    proportion test.

    Health monitoring requires some state information related
    to the noise source to be maintained. The tests should be designed
    in a way that a specific number of samples guarantees a state
    flush (no hung states). We suggest flush size $W \leq 1024$ to
    match with the NIST SP 800-90B required tests (See Section 4.4 in
    \cite{TuBaKe+18}). The state is also fully zeroized in a system reset.

    \underline{Rationale:}
    The two mandatory tests can be built with minimal circuitry.
    Full histograms are not required, only simple counter registers:
    repetition count, window count, and sample count.
    Repetition count is reset every time the output sample value
    changes; if the count reaches a certain cutoff limit, a noise alarm
    (\verb|BIST|) or failure (\verb|DEAD|) is signaled. Window counter is
    used to save every $W$'th output (typically $W \in { 512, 1024 }$.)
    The frequency of this reference sample in the following window is
    counted; cutoff values are defined in the standard. We see that the
    structure of the mandatory tests is such that, if well implemented,
    no information is carried beyond a limit of $W$ samples.

    Section 4.5 of \cite{TuBaKe+18} explicitly permits additional
    developer-defined tests and several more were defined in early
    versions of FIPS 140-1 before being ``crossed out.'' The choice
    of additional tests depends on the nature and implementation of the
    physical source.

    Especially if a non-cryptographic a conditioner is used in hardware,
    it is possible that the AIS 31 \cite{KiSc11} online tests are
    implemented by driver software. They can also be implemented in hardware.
    For some security profiles AIS 31 mandates that their tolerances are
    set in a way that the probability of an alarm is at least $10^{-6}$
    yearly under ``normal usage.'' Such requirements are problematic
    in modern applications since their probability is too high for
    critical systems\footnote{Currently (2020) about $10^{10}$ secure
    elements are shipped yearly, many in critical applications and with
    TRNGs, according to \url{https://www.eurosmart.com}.}.
    There rarely is anything that can or should be done about a non-fatal
    alarm condition in an operator-free, autonomous system. However,
    AIS 31 allows the DRBG component to keep running despite a failure in
    its Entropy Source, so we suggest re-entering temporary \verb|BIST|
    state (Section \ref{sec:security-controls}) to signal a non-fatal
    statistical error if such (non-actionable) signaling is necessary.
    Drivers and applications can react to this appropriately (or simply
    log it) but it will not directly affect the availability of the TRNG.
    A permanent error condition should result in \verb|DEAD| state.

\subsubsection{Non-cryptographic Conditioners}
\label{sec:noncrypto}

    As noted in Section \ref{sec:intro-cond}, physical randomness sources
    generally require a post-processing step called \emph{conditioning} to
    meet the desired quality requirements, which  are outlined in Section
    \ref{sec:req-es}.

    The approach taken in this interface is to allow a combination of
    non-cryptographic and cryptographic filtering to take place. The
    first stage (hardware) merely needs to be able to distill the entropy
    comfortably above 4 bits per byte (Sect. \ref{sec:req-entropy},
    Entropy) and to guarantee that the samples are independent
    (Sect. \ref{sec:req-iid}, IID).

    \begin{itemize}
    \item   One may take a set of bits from a noise source and XOR them
            together to produce a less biased (and more independent) bit.
            If the source model is well understood, such a construction
            lends itself well to analysis and entropy estimation \cite{Da02}.
    \item   The von Neumann extractor \cite{Ne51} looks at consecutive
            pairs of bits, rejects 00 and 11, and outputs 0 or 1 for
            01 and 10, respectively. It will reduce the number of bits to
            less than 25\% of original but the output is provably unbiased
            (assuming independence).
    \item   Blum's extractor \cite{Bl86} can be used on sources
            whose behavior resembles $n$-state Markov chains. If its
            assumptions hold, it also removes dependencies, creating an IID
            source.
    \item   Other linear and non-linear correctors such as those
            discussed by Dichtl and Lacharme \cite{La08}.
    \end{itemize}

    Note that the hardware may
    also implement a full cryptographic conditioner to in the entropy
    source, even though the software driver still needs
    a cryptographic conditioner too (Sect. \ref{sec:req-state}).

    \underline{Rationale:}
    The main advantage of non-cryptographic filters is in their
    energy efficiency, relative simplicity, and amenability to mathematical
    analysis. If well designed, they can be evaluated in
    conjunction with a stochastic model of the noise source itself.
    They do not require computational hardness assumptions.

    In some cases, an entropy source (and the circuit that implements it)
    may have a uniquely identifiable hardware ``signature.'' This can be
    harmless or even useful in some applications (as random sources may
    exhibit PUF-like features) but highly undesirable in others (anonymized
    virtualized environments and enclaves).

    Such virtualized environments are probably better off just using
    \verb|/dev/urandom| of the host rather than sharing the host's
    hardware-backed Entropy Source to the guest environment. Also note the
    source entropy requirement (Sect. \ref{sec:req-es}, Secret State)
    when sharing such generators.

\subsubsection{Cryptographic Conditioners}
\label{sec:crypto-cond}

    Cryptographic conditioners are always required on the software side of
    the PollEntropy ISA boundary. They may be also implemented on the
    hardware side if necessary. In any case, the PollEntropy output must
    always be compressed 2:1 (or more) before being used as keying material
    or considered ``full entropy.''

    Examples of cryptographic conditioners include the random pool
    of the Linux operating system, secure hash functions (SHA-2/3,
    SHAKE \cite{NI15,NI15A} ),
    and the AES-based CBC-MAC construction of SP 800-90B \cite{TuBaKe+18}.

    In some constructions, such as the Linux RNG and SHA-3/SHAKE \cite{NI15}
    based generators the cryptographic conditioning and output (DRBG)
    generation is provided by the same component.

    \underline{Rationale:}
    For many low-power targets constructions such as Intel's
    \cite{Me18} and AMD's \cite{AM17} hardware AES CBC-MAC conditioner
    would be too complex and expensive to implement solely to serve
    \mnemonic{pollentropy}. On the other hand, simpler non-cryptographic
    conditioners may be too wasteful on input entropy if very high-quality
    random output is required -- ARM TrustZone TRBG \cite{AR17} outputs
    only 10Kbit/sec at 200 MHz. Hence a resource-saving compromise is
    made between hardware and software generation that allows an
    implementation to use the RISC-V cryptographic ISA.

%   Even if a DRBG seed obtains a sufficient amount of entropy in total,
%   some bits may be more important than others. For example, the IV values
%   of a counter-mode DRBG are less important than the key bits;
%   if an adversary knows the key bits then the IV (counter value) is
%   easy to determine. The inverse is not true. Cryptographic
%   conditioning is required to spread the entropy across all bits.


\subsubsection{The Final Random: DRBGs}
\label{sec:drbgs}

    All random bits reaching end users and applications must come from a
    cryptographic DRBG. These are generally implemented by the driver
    component in software. The RISC-V AES and SHA instruction set extensions
    should be used if available, since they offer additional
    security features such as timing attack resistance.

    Currently recommended DRBGs are defined in NIST SP 800-90A (Rev 1)
    \cite{BaKe15}: \verb|CTR_DRBG|, \verb|Hash_DRBG|, and \verb|HMAC_DRBG|.
    Certification often requires known answer tests (KATs) for the symmetric
    components and the DRBG as a whole. These are significantly easier to
    implement in software than in hardware. In addition to the directly
    certifiable SP 800-90A DRBGs, a Linux-style random pool construction
    based on ChaCha20 \cite{Mu20} can be used, or an appropriate construction
    based on SHAKE256 \cite{NI15}.

    These are just recommendations; programmers can adjust the usage of the
    CPU Entropy Source to meet future requirements.


\subsection{Quantum vs Classical Random}
\label{sec:quantum}

    \begin{quote}
        {\it ``The NCSC believes that classical RNGs will continue to
        meet our needs for government and military applications for the
        foreseeable future.''}
        \flushright -- U.K. QRNG Guidance, March 2020 \cite{NC20}.
    \end{quote}

    A Quantum Random Number Generator (QRNG) is a TRNG whose source of
    randomness can be unambiguously identified to be a \emph{specific}
    quantum phenomenon such as quantum state superposition, quantum state
    entanglement, Heisenberg uncertainty, quantum tunneling, spontaneous
    emission, or radioactive decay \cite{IT19}.

    Direct quantum entropy is theoretically the best possible kind of
    entropy. A typical TRNG based on electronic noise is also largely
    based on quantum phenomena and is equally unpredictable - the difference
    is that the relative amount of quantum and classical physics involved is
    difficult to quantify for a classical TRNG.

    QRNGs are designed in a way that allows the amount of quantum-origin
    entropy to be modeled and estimated. This distinction is important in
    the security model used by QKD (Quantum Key Distribution) security
    mechanisms which can be used to protect the physical layer (such as
    fiber optic cables) against interception by using quantum mechanical
    effects directly.

    This security model means that many of the available
    QRNG devices do not use cryptographic conditioning and may fail
    cryptographic statistical requirements \cite{HuHe20}. Many implementers
    may consider them to be entropy sources instead.

    Relatively little research has gone into QRNG implementation security,
    but many QRNG designs are arguably more susceptible to leakage than
    classical generators (such as ring oscillators) as they tend to employ
    external components and mixed materials.

    \paragraph{Post-Quantum Cryptography.}
    The classical/quantum origin of randomness is not important in NIST
    Post-Quantum Cryptography (PQC) \cite{NI16}. Recall that cryptography
    aims to protect the confidentiality and integrity of data itself
    and does not place any requirements on the physical communication
    channel (like QKD). Classical good-quality TRNGs are perfectly suitable
    for generating the secret keys for PQC protocols that are hard for
    quantum computers to break, but implementable on classical computers.
    What matters in cryptography is that the secret keys have enough true
    randomness (entropy) and that they are generated and stored securely.

    Of course one must avoid DRBGs that are based on problems that are
    easily solvable with quantum computers, such as factoring \cite{Sh94}
    in the case of Blum-Blum-Shub generator \cite{BlBlSh86}. However
    most symmetric algorithms are less affected as the best quantum
    attacks are still exponential to key size \cite{Gr96}.
%    (rather than polynomial
%   as is the case for RSA and Discrete Logarithms).

    As an example, the original Intel RNG \cite{Me18}, whose output
    generation is based on AES-128 can be attacked using Grover's algorithm
    with approximately square-root effort \cite{JaNaRo+20}.
    While even ``64-bit'' quantum security is extremely difficult to
    break, many applications specify a higher security requirement.
    NIST \cite{NI16} defines AES-128 to be ``Category 1'' equivalent
    post-quantum security, while AES-256 is ``Category 5'' (highest).
    We avoid this possible future issue by exposing a more direct access
    to the entropy source, which can derive its security from
    information-theoretic assumptions only.

%    This eliminates potential problems caused by computational hardness
%   assumptions.


