
A new machine-mode {\bf WARL} CSR is defined at
address $XXX$
called \mnemonic{mcrypto}
which contains bits indicating the presence or absence of Cryptography
Extension features.

\begin{figure}[h]
\vspace{0.5cm}
\centering
\begin{bytefield}[endianness=big,bitwidth=1.4em]{32}
\bitheader{0-31} \\
\bitbox{27}{Reserved / WARL}
\bitbox{1}{{\tt Sm}}
\bitbox{1}{{\tt Sn}}
\bitbox{1}{{\tt Sb}}
\bitbox{1}{{\tt V }}
\bitbox{1}{{\tt R }}
\\
\end{bytefield}
\vspace{-0.5cm}
\caption{
Fields of the \mnemonic{mcrypto} CSR.
}
\end{figure}

\begin{enumerate}
\item[{\tt R} -]
    Indicates the Random Bit Generation extension is implemented.

\item[{\tt V} -] 
    Indicates the vector cryptography extensions are implemented.
    Note that this bit implies the presense of the Vector extension.

\item[{\tt Sb} -]
    Indicates the scalar cryptography extensions base subset.
    These are the instructions which overlap with the Bitmanip
    extension, and the LUT4 instruction.

\item[{\tt Sn} -]
    Indicates the scalar SHA2 and AES instructions are implemented.

\item[{\tt Sm} -]
    Indicates the scalar SM3 and SM4 instructions are implemented.
\end{enumerate}

Reserved bits must return zero when read, and will be used to indicate
new features in future versions of the Cryptographic Extension.

If {\em any} part of the scalar cryptography extension is implemented
(that is, reading the \mnemonic{mcrypto} CSR returns a non-zero value),
then the
{\tt K}
bit of the \mnemonic{misa} register must be set\footnote{
    Because {\tt K} is for {\tt K}rypto and {\tt C} was taken.
}.
If the
{\tt K}
bit of the \mnemonic{misa} register is clear, this
implies the \mnemonic{mcrypto} register will return zero when read.

\todo{
Where to put multi-precision arithmetic? Current proposals are likely
to be completely re-written.
Where to put indexed load store? Would these be better as a separate
extension?
}
