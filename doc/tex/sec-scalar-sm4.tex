
\newpage
\subsection{Scalar SM4 Acceleration}
\label{sec:scalar:sm4}

This section proposes acceleration instructions for
the SM4 block cipher\cite{block:sm4:1, ietf:sm4}.

\begin{cryptoisa}
RV32 / RV64:
    ssm4.ed     rd, rs1, rs2, bs
    ssm4.ks     rd, rs1, rs2, bs
\end{cryptoisa}

The instructions are taken from proposals found in \cite{MJS:LWAES:20}.
They are very lightweight and require only a single SBox instance.
They are designed to give an very high performance improvement with
minimal area requirements, and resemble a TTables-esq
software implementation.

\begin{itemize}
\item \mnemonic{ssm4.ed} - Encrypt/Decrypt instruction. Applies the
    SBox and $L$ transformations as part of the round function.
\item \mnemonic{ssm4.ks} - KeySchedule instruction. Applies the
    SBox and $L'$ transformations as part of the KeySchedule.
\end{itemize}

\begin{figure}[h]
\begin{lstlisting}[language=pseudo]
ssm4.ed(rs1, rs2, bs):
    x.8   = rs2.8[bs]
    x.32  = zext(SM4SBox(x.8), 32)
    x.32  = x.32 ^ (x.32 << 8) ^ (x.32 << 2) ^ (x.32 << 18) ^
                   ((x.32 & 0x3F) << 26) ^ ((x.32 & 0xC0) << 10)
    x.32  = ROTL(x.32, 8*bs)
    rd    = x.32 ^ rs1

ssm4.ks(rs1, rs2, bs):
    x.8   = rs2.8[bs]
    x.32  = zext(SM4SBox(x.8), 32)
    x.32  = x.32 ^ ((x.32 & 0x07) << 29) ^ ((x.32 & 0xFE) << 7) ^
                   ((x.32 & 1) << 23) ^ ((x.32 & 0xF8) << 13)
    x.32  = ROTL(x.32, 8*bs)
    rd    = x.32 ^ rs1
\end{lstlisting}
\caption{Scalar SM4 instruction pseudocode.}
\end{figure}

\note{
The SM4 and RV32 AES (see Section \ref{sec:scalar:aes:rv32})
instructions are designed to share much of their data-path, and
can be implemented with much logic shared  between their SBoxes
in particular.
}
