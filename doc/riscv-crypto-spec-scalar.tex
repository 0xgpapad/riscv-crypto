\documentclass[11pt]{article}
\usepackage{riscv-crypto-spec}

\title{RISC-V Cryptographic Extension Proposals\\Volume I: Scalar \& Entropy Source Instructions}
\author{Editor: Ben Marshall\\ben.marshall@bristol.ac.uk}
\date{Version $0.7.2$ (\today) \\
\medskip
\url{\repourl{}} \\
\href{\repourl{}}{
{\small \tt git:\import{../build/spec/}{spec.commit}}}
}

\begin{document}

% ============================================================================

\maketitle

\import{./tex/}{contributors.tex}

\tableofcontents

% ============================================================================

\newpage
\section{Introduction}
\label{sec:intro}
\import{./tex/}{sec-scalar-intro.tex}

\newpage
\section{Implementation Profiles}
\label{sec:profiles}
\import{./tex/}{sec-scalar-profiles.tex}

\newpage
\section{Scalar Extension}
\label{sec:scalar}

As per the RISC-V Cryptographic Extensions Task Group charter:
``{\em The committee will also make ISA extension proposals for lightweight
scalar instructions for 32 and 64 bit machines that improve the performance
and reduce the code size required for software execution of common algorithms
like AES and SHA and lightweight algorithms like PRESENT and GOST}."

\bigskip

For context, some of these instructions have been developed based on academic
work at the University of Bristol as part of the XCrypto project
\cite{MPP:19},
and work by
Paris T\'{e}l\'{e}com on acceleration of lightweight block ciphers
\cite{TGMGD:19}.


% ============================================================================

\import{./tex/}{sec-scalar-bitmanip.tex}
\import{./tex/}{sec-scalar-aes.tex}
\clearpage
\import{./tex/}{sec-scalar-sha2.tex}
\import{./tex/}{sec-scalar-sm3.tex}
\import{./tex/}{sec-scalar-sm4.tex}
\clearpage
\import{./tex/}{sec-scalar-timing.tex}


\newpage
\section{Entropy Source Extension}
\label{sec:randombit}
\import{./tex/}{sec-entropy-source.tex}

% ============================================================================

\newpage
\printbibliography

% ============================================================================

%
% Appendix items with annotations, benchmarking and experimental
% evidence etc.
%

\newpage
\begin{appendices}
\label{sec:appendix}
\import{./tex/}{appx-scalar-encodings.tex}
\import{./tex/}{appx-entropy.tex}
\import{./tex/}{appx-materials.tex}
\end{appendices}

% ============================================================================

\end{document}

